%% LaTeX2e class for student theses
%% sections/conclusion.tex
%%
%% Karlsruhe University of Applied Sciences
%% Faculty of Computer Science and Business Information Systems
%%
%% --------------------------------------------------------
%% | Derived from sdqthesis by Erik Burger burger@kit.edu |
%% --------------------------------------------------------


\chapter{Resume and Outlook}
\label{ch:Resume and Outlook}

Summarizing the results achieved in the development of a comprehensive reservation system for the management of charging infrastructure, this final chapter provides a resume concluding the tasks of this work. 
Furthermore, based on the achieved results, a consideration of the contribution to the proposals, elaborated in the literature, is given. 
Together with these retrospective views, this chapter concludes by providing an outlook on future possibilities for this category of system and proposing directions for future work. \\
%% --------------------------------------------------------
%% Resume
%% --------------------------------------------------------
\noindent Especially, the administration and maintenance of charging infrastructure face several constraints, that are not determined by existing standards or technical limitations.
A frequently underestimated fact refers to the process--related impediments to ensure the intended behavior from the system, the charging environment, and the driver.
Beyond the behavior of the individual system parts and the corresponding actors, the exchange of information between these components is another aspect, that needs to be considered in the process of designing these types of systems. 
In order to provide a sophisticated experience, it should allow interconnectivity with stations from several manufacturers, interoperability with existing services, and ease of use by users.
Therefore, the proposed design and development of a comprehensive solution aims to address all these concerns, by extending the aforementioned \acrshort{ocpp} communication standard.
By creating this \acrshort{poc}, the proposal should demonstrate, that it is feasible to coordinate charging session bookings for immediate as well as future sessions, by incorporating a basic set of functionalities.
Concerning the fulfillment of these goals, initially stated at the beginning of this work, the resulting design proposal is at least capable of handling charging appointments, according to the \acrshort{ocpp} standard in version 1.6. 
Thus, it is applicable for being implemented within a real--world scenario, dealing with the problem of coordinated \acrshort{cs} allocation, presupposed the stations support the version mentioned above of the standard. 
In order to create the bookings, the system offers besides the standard \textit{ReserveNow} method, which only permits immediate blocking of a connector, an extension of the \acrshort{ocpp} functionality, to allow users to make arrangements in advance.
Enabling \acrshortpl{evu} to plan future charging sessions and recurring bookings, is not accommodated by other system approaches at the time of this research, as far as the author of this work is concerned.
By using this solution, it is possible for drivers to make reservations over an extended period of time, in a more convenient manner. This proposal alleviates the inconvenience of making a booking for every single day the car needs to be charged.
Concerning the elaborated scenario, this also caters to the requirements of the stakeholders. 
Particularly in the suggested situations this study considers, where designated user groups have specific parking spots they are associated with, which offer only limited available charging points.
In a broader context, these scenarios could also apply to car parks, owned by providers or companies, who need to explicitly regulate the individual parking spaces, equipped with \acrshortpl{cs}.
The range of domains, this proposal is applicable to, in order to allow advanced reservation life cycle management, as well as a more detailed way of monitoring, is diverse.
Thus, the author assumes, that this approach, in terms of contributing to the existing research for \acrshort{cs} management, at least gives a better understanding of the obstacles as well as the opportunities, by introducing such solutions. \\
%% --------------------------------------------------------
%% Scientific Contribution
%% --------------------------------------------------------
\noindent Taking into account the above--listed aspects, the reviewed proposal for a systematic approach for \acrshortpl{cs} allocation allows leastwise the following contributions in terms of supporting the existing research.
In order to extend the range of scenarios considered in other projects, the introduction of further requirements, especially in terms of background processes broadens the ratio of possible unintended behavior. 
This for example includes the mitigation techniques, such as managing non--arriving \acrshortpl{evu} and initiating mitigation processes, prior to the start of a charging session at stations with pending reservations.
Alongside highlighting problematic situations, this documentation also presents suggestions for resolving such conflicts. 
For this purpose, it solely considers the options typically available to such systems, without the need for external factors, mostly not accessible in other scenarios.
This ensures a higher degree of enforcing arriving reservations and a more automated approach, to managing infrastructure availability in a generic way. 
Furthermore, it avoids unnecessary blockages, fostered by the background processes mentioned above. 
Combined with the additional life cycle states, describing the validity of a booking, a detailed consideration of relevant error cases, is elaborated.
Besides the exceptions from the designed processes, this collection of possible misbehavior, includes the potential errors produced by the charging infrastructure as well. 
Using this new terminology in conjunction with the understanding of the edge cases that have already been identified, there is potential for elaborating scenarios that require more complex combinations of features.
This allows further extensions to be developed in combination with the elaborated entities, which serve as a guideline for encapsulating the standardized request objects. \\
%% --------------------------------------------------------
%% Outlook
%% --------------------------------------------------------
\noindent Switching the perspective from solely focusing on managing stations or infrastructure at a specific location. 
By considering the possibility of connecting different infrastructures to create a network of interconnected stations, communicating via a common backbone, the problem most highlighted in the context of \acrshort{emobility}, also known as 'range anxiety' \cite{rauh_understanding_2015}, could also be addressed.
As already suggested in the literature by \cite{zarkeshev_charging_2018}, the use of roaming capabilities, to build a heterogenous network of \acrshortpl{cs}, may be capable of alleviating such fears.
Through the enablement of \acrshortpl{ev} to share data on their current battery status and charging needs directly with the station, the ability to reserve charging sessions along the route would be possible.
Without the intervention of the driver, the car should be capable of planning its charging sessions, based on the given destination and guarantee its owner the possibility to recharge.
Especially during longer trips, as addressed by 'range anxiety', and an unknown environment, this offers more convenience in using \acrshortpl{ev}. 
Moreover, linking several \acrshortpl{cso}, \acrshortpl{cpo} and \acrshortpl{emsp} would lead to a higher acceptance of their services by the \acrshortpl{evu}, and thus to an increase in both usage and profits.
This facilitates both the development of new areas and offers numerous business prospects, by supporting the installation of stations and expanding the existing infrastructure.
Nevertheless, it is important to take into account the increased load on the grid, that would be caused as a result. 
However, by knowing the amount of time, that the \acrshort{ev} will eventually park at the \acrshort{cs} location, combined with the current battery state, the amount of energy required to recharge to a full state of charge can be determined.
The remaining unneeded power can be used through smart charging scenarios such as \acrshort{v2g}, which aims to support overall grid stability using excessive power from \acrshortpl{ev}.
Other applications could include \acrshort{v2b} to return energy to the building, where the power unit is located, \acrshort{v2h} to replace the building by the driver's home, or \acrshort{v2x} to cover various targets.
In order to further promote the adoption of \acrshortpl{ev}, enhance the corresponding infrastructure, and provide support for all the scenarios mentioned, the need for future development work in this area is obvious. \\
%% --------------------------------------------------------
%% Future Work
%% --------------------------------------------------------
\noindent Regarding the purpose of subsequent elaborations, below are some examples that the author of this work considers significant and possibly useful for future research. 
First to be mentioned, are the fundamental constraints, imposed by the state of the grid, which require special attention. 
Without a functional grid to power the relevant stations and vehicles, further developments in energy storage and infrastructure observation are pointless.
Therefore, the simulations already in existence, taking into account the different types of elaborated reservations to measure their impact on the journey duration and overall comfort could be used.
Introduced by Basmadjian et al. in \cite{basmadjian_reference_2020,basmadjian_interoperable_2019}, these results could be used as groundwork, to create scenarios specifically for simulating the effects of reservations, to cushion peek loads using \acrshort{v2g}.
Utilizing the existing software tooling like \href{https://github.com/aicenter/agentpolis}{\textit{AgentPolis}} \cite{noauthor_agentpolis_2022}, used in the above simulations to create the infrastructure and agent--based drivers, all that needs to be added is the power grid.
Alongside the effective prevention of parking space blockades by users without bookings, using sensor--based solutions, the consideration of mobile \acrshortpl{cs}, as movable units, requires further investigations. 
Based on the existing literature, only loose papers take this type of station into account for application within reservation scenarios. 
Abstracting from the issues of continuous power demand from stationary units, mobile stations in the form of vehicles could potentially address the problems of limited parking spaces equipped with \acrshortpl{cs}. 
Considering the integration of reservations into more mobile scenarios can offer new opportunities, as already examined by \cite{zhang_mobile_2020}.
Putting all these deliberations in a nutshell, the field of reservation systems for charging infrastructure management offers many opportunities and approaches for improvement. 
Apart from the creation of new standards, to introduce another more unified communication protocol, the use of machine learning algorithms, for extended optimization techniques, salvages huge potential as well. \\
%% --------------------------------------------------------
%% Results
%% --------------------------------------------------------
The concrete results of this master's thesis and the necessary changes made to the applications during this work are available in the corresponding public forks of the original repositories, within the '\textit{reservation--process}' branch. 
The relevant \acrshortpl{url} as hyperlinks, are deposited in the following declarations for the \href{https://github.com/JulianHBuecher/ev-server/tree/reservation-process}{\textbf{backend}}, \href{https://github.com/JulianHBuecher/ev-mobile/tree/reservation-process}{\textbf{mobile}} and \href{https://github.com/JulianHBuecher/ev-dashboard/tree/reservation-process}{\textbf{web}} components of the \textit{Open e--Mobility} project. 
