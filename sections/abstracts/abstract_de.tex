%% LaTeX2e class for student theses
%% sections/abstracts/abstract_de.tex
%%
%% Karlsruhe University of Applied Sciences
%% Faculty of Computer Science and Business Information Systems
%%
%% --------------------------------------------------------
%% | Derived from sdqthesis by Erik Burger burger@kit.edu |
%% --------------------------------------------------------

\Abstract
% Problem
Die sich stetig weiterentwickelnden Technologien im Kontext von Elektromobilit\"at  f\"uhren unweigerlich zu einer erh\"ohten Auslastung von Lade--, wie auch Verkehrsinfrastruktur. Dabei sind Engp\"asse in der Verwendung von Ladestationen unvermeidbar. 
Trotz der Nutzung intelligenter Mechanismen zur Lastverteilung und computer-gest\"utzter Verfahren f\"ur das Laden elektrisch-betriebener Fahrzeuge, wie dem Smart Charging, finden Ans\"atze wie das gezielte Verwalten von Ladeinfrastruktur durch Reservierungen nur wenig Verwendung.
Die daraus resultierende Notwendigkeit eines systemgest\"utzten Ansatzes zur Verwaltung von Ladem\"oglichkeiten ist somit ein zentraler Aspekt f\"ur die Etablierung von elektrischen Fahrzeugen innerhalb der Gesellschaft und deren Akzeptanz als verl\"assliches Fortbewegungsmittel.\\
% Target
Welche funktionalen- und nicht-funktionalen Anforderungen sollte solch ein System erf\"ullen? Diese Frage wird anhand konkreter Use Cases, eingebettet in ein realitätsnahes Szenario, in der hier vorliegenden Master--Thesis in Kooperation mit der SAP~SE aus unterschiedlichen Perspektiven beleuchtet.\\
% Approach
Zu Beginn stand eine theoretische Betrachtung der bereits vorhandenen Funktionalit\"aten, die basierend auf offenen Industriestandards, wie dem \acrfull{ocpp} oder dem \acrfull{ocpi}, durch die Hersteller elektrischer Lades\"aulen bereits angeboten werden und durch Ber\"ucksichtigung die Interoperabilit\"at der L\"osung erlauben.
Anhand dieser standardisierten Operationen, erfolgte anschließend eine Sammlung von Anforderungen, die w\"ahrend der Bearbeitungszeit der hier vorliegenden Arbeit noch nicht als Teil einer der oben genannten Standards ber\"ucksichtigt wurden. Jedoch im Kontext dieser Arbeit als essenziell f\"ur das zu erreichende Ziel identifiziert werden konnten.
Daraus resultierte wiederum durch Selektion einer Teilmenge notwendiger Basisfunktionalit\"at eine Umsetzung im Rahmen eines \acrfull{poc}, der die Umsetzbarkeit des vorgeschlagenen Ansatzes demonstriert.\\
% Results
Nach Durchlaufen der theoretischen und praktischen Phase konnte eine Grundmenge an Operationen und Vorg\"angen synthetisiert und als Teil der SAP~SE Open Source Software \textit{Open e-Mobility} beispielhaft unter Verwendung des konstruierten Szenarios umgesetzt und anhand einer simulierten Testumgebung validiert werden.
In Anlehnung an bestehende Reservierungssysteme demonstriert dieser Ansatz, integriert in der webbasierten \href{https://github.com/JulianHBuecher/ev-dashboard/tree/reservation-process}{Oberfl\"ache}, der \href{https://github.com/JulianHBuecher/ev-mobile/tree/reservation-process}{mobilen Applikaktion}, sowie des \href{https://github.com/JulianHBuecher/ev-server/tree/reservation-process}{Backend--Servers}, eine Art und Weise, wie zuk\"unftige Nutzer \"uber das System zugewiesene Ladepunkte wie auch deren Reservierungen verwalten k\"onnen.\\
% Further Research Areas
Entsprechend der zugrundeliegenden Implementierung erlaubt das Konzept der Reservierung den Einsatz innerhalb weiterer Szenarien, wie beispielsweise im Kontext von Smart Charging. Hier sind in Kombination mit den Zeitr\"aumen für die einzelnen Reservierungen Ans\"atze bez\"uglich bidirektionaler Ladevorg\"ange, dem \acrfull{v2g}, zur Reduktion der Auslastung des Stromnetzes durchaus denkbar.
