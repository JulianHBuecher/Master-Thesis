%% LaTeX2e class for student theses
%% sections/abstract_en.tex
%%
%% Karlsruhe University of Applied Sciences
%% Faculty of  Computer Science and Business Information Systems
%%
%% --------------------------------------------------------
%% | Derived from sdqthesis by Erik Burger burger@kit.edu |
%% --------------------------------------------------------


\Abstract
\noindent Based on the advances in battery technology and the increasing number of electronic vehicles (EV) on the roads, the necessary infrastructure such as charging stations and the underlying power grid becomes a bottle neck for electric vehicle drivers. Despite batteries with longer battery life, their operating distance is still far behind equivalent vehicles with combustion engines and the process of recharging the battery is not comparable with replenishment \cite{orcioni_electric_2018} via fossil fuels. To counteract this problem, in addition to scaling physical charging infrastructure \cite[p.~165231]{adib_e-mobility_2019}, it is necessary to implement technology-supported processes for managing existing facilities.
\noindent These processes are intended to enable drivers having a guaranteed option for charging their vehicles, which results in an increase of reliability for planning their routes. Implementations of this functionality could be performed by establishing a reservation system including a user management for creating linkages between a reservation and the driver with the corresponding vehicle.\\
\noindent For this purpose, design approaches of interoperable reservation systems for the public use of charging stations have already been evaluated in the literature \cite{basmadjian_interoperable_2019,adib_e-mobility_2019}. As part of these analyses, a distinction between four different procedures for reserving the available loading infrastructure carried out. Beside "Uncertain Ad-Hoc", "Guaranteed Ad-Hoc", "Uncertain Planned" the authors distinguished into "Guaranteed Planned (Full)" \cite{basmadjian_reference_2020} for describing their processes. In the subsequent case study, only the "Uncertain Ad-Hoc" procedure was implemented using the E-Mobility System Architecture (EMEA) \cite{kirpes_e-mobility_2019} as proof of concept, which considers the linking between user and the booked charging infrastructure. 
The investigation did not include the handling of exceptions such as the management of misplaced vehicles, unauthorized parking beyond the booking period, and possible delays in starting the loading process. Moreover the reservation scenarios like "Uncertain Planned" or "Guaranteed Planned (Full)" were not included consequently reservations for journeys planned for a specific time in the future are not mapped in the given examples. In case of business customers, which needs a charging station for a  specific time in the future, e.g. in the context of a business trip, they have to rely on the chance of a free charger.\\
\noindent As the subject of this work, a unified process will be established providing the possibility to manage and plan a reservation to fill the gap in the existing conceptual reservation processes. The underlying problem statement will consider the requirements given in a charging scenario for drivers of a company internal EV fleet and potential customer visits and addresses the missing "planned" scenarios in the literature. 
In contrast to the work mentioned above, the charging infrastructure will be described as a closed ecosystem in the context of a company, which is not be able to use by external firms or citizens.
The preceding design process includes an assessment of the technical feasibility based on the existing technology stack of the open source software "Open E-Mobility", which is intended to enable intelligent management of charging infrastructure utilising the OCPP standard version 1.6 \cite{noauthor_ocpp_nodate}, as well. Due to the missing mapping of a reservation by the specified solution, the resulting design will be implemented as a prototype with a proof of feasibility and evaluated in view of its usability. Therefore, the existing backend service, its corresponding graphical user interface (GUI) for administrative tasks and the client frontend will be used as starting point for the development. Furthermore, adjustments to the commonly used data model in the preexisting services will be part of the implementation process enabling the unified reservation process. Nevertheless, aspects like backwards compatibility and breaking changes will be considered.