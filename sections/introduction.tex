%% LaTeX2e class for student theses
%% sections/introduction.tex
%%
%% Karlsruhe University of Applied Sciences
%% Faculty of  Computer Science and Business Information Systems
%%
%% --------------------------------------------------------
%% | Derived from sdqthesis by Erik Burger burger@kit.edu |
%% --------------------------------------------------------

\chapter{Introduction}
\label{ch:Introduction}

The declining availability of fossil fuels for the manufacturing of products such as petrol and the resulting consequences like global warming by burning these energy carriers are only two of the arguments driving the debate to promote the development of more sustainable behavior in different areas of daily life \cite{kathiresh_e-mobility_2022}.
Besides switching from coal or oil as a viable source for energy generation to renewables, or the transition from cars powered by \acrfull{ice} to a more environmentally friendly alternative, is an essential step in reducing the amount of \acrfull{co2} in the atmosphere and slow down the accelerating speed of the climate change.
As part of the process of 'electrification of energy conversion systems in mobile platforms' \cite[165226]{adib_e-mobility_2019} in the context of \acrfull{emobility}, the industry and users still have some challenges to deal with.
The most significant issues are the inadequacies in key technologies such as batteries or power trains. Compared to their fossil fuel alternatives, they are less efficient and durable, resulting in limited operating ranges due to their lower capacity. 
In combination with the current shortcomings, the lack of charging facilities in sparsely populated areas or cities is another unaddressed problem that adversely affects the acceptance of \acrshortpl{ev}.
Other negative consequences, such as the phenomenon of 'range anxiety' \cite{rauh_understanding_2015}, experienced by many \acrfullpl{evu} in rural areas lacking a widespread charging infrastructure, additionally affect the acceptance and conversion of cars with \acrshort{ice} to \acrfull{hev} or \acrfull{fev}. 
Furthermore, the waste generated by the production or disposal of electric vehicles is a crucial factor. In particular, the disposal of lithium-ion batteries has a harmful impact on the surrounding environment \cite{xu_generation_2017}, which must be taken into account when further developing this technology.
Nevertheless, the long-term benefits of \acrshortpl{ev} and their impact on reducing \acrshort{co2} and the use of fossil fuels are undeniable.
Because of this, numerous institutions and governments are eager to endorse the growth of electric mobility and try to counteract the negative impacts mentioned above.
For example, government subsidies, as well as other incentives offered by manufacturers, are measures intended to promote \acrshortpl{ev} as a viable means of transport in the future.

\section{Conceptual Formulation}
\label{ch:Introduction:sec:Conceptual Formulation}

Due to the requirement for a system to administer and monitor the associated infrastructure, different approaches for managing \acrshortpl{cs} and the corresponding connectors have emerged.
As a result, multiple software companies partnered with charging station manufacturers or standardization bodies to establish a common denominator for information exchange.
Their work led to the creation of standards such as the \acrfull{ocpp}, \acrfull{ocpi} and ISO 15118, which offer a comprehensive range of features for communication between the \acrshortpl{evu}, \acrshortpl{ev}, and their respective \acrshortpl{cs}.
Despite the extensive range of processes covered, the mentioned standards are not considered complete and require further refinement and enhancements for special situations.
Therefore, almost every year, new major versions with minor fixes and improvements in the form of new features are released. 
Considering certain feature gaps, particularly management or administrative processes to ensure fair and regulated use of charging options, functionalities such as the aforementioned are often missing in current implementations. 
Addressing this need, this document describes a process for reserving charging points or connectors in advance for a specified period of time.
Besides providing the \acrshortpl{evu} with a guarantee for charging, this feature should facilitate a detailed administration of \acrshortpl{cs} for owners of semi-public or private parking spaces and offer an alternative to the current first-come-first-serve principle in a more regulated way.

\section{SAP SE}
\label{ch:Introduction:sec:SAP SE}

This work is part of a master's thesis at SAP SE, a multinational software company based in Walldorf, Germany. 
It is one of the world's leading enterprise software companies, renowned for its innovative solutions that help businesses manage their operations effectively. 
Founded in 1972 by Dietmar Hopp, Hasso Plattner, Claus Wellenreuther, Hans-Werner Hector, and Klaus Tschira \cite{noauthor_inventing_nodate}, the company aims to develop standard application software for real-time business data processing and to improve the way businesses manage their operations. 
Apart from \textit{SAP S/4HANA} and \textit{SAP ERP} \cite{noauthor_sap_nodate}, SAP offers a wide range of enterprise software products and services to various business needs and industries, such as manufacturing, finance, healthcare, retail or public sector organizations.
Furthermore, SAP is engaged in developing emerging technologies such as artificial intelligence, machine learning, the Internet of Things (IoT), or solutions in the \acrshort{emobility} sector.
This includes software for managing existing \acrshortpl{cs} within one company or applications to support the \acrshortpl{evu} in finding free charging opportunities more easily. 
As part of this approach to drive the expansion of \acrshort{emobility} in the enterprise world, this thesis provides additional concepts and example workflows for the advanced management of charging infrastructure. 

\newpage

\section{Document Structure}
\label{ch:Introduction:Document Structure}

As the main focus of this work, the following chapters describe a potential solution for conceptualization and implementation of a reservation system based on the existing SAP open source solution, \textit{Open e-Mobility}. 
Starting with the gathering of essential contextual information, the subsequent conceptual Section identifies the necessary use cases for the design of the corresponding processes. 
Followed by a structured timeline describing the approaches for both the theoretical part and the implementation part, broken down into steps. 
Next, an assessment of previous research investigating comparable methods for addressing the primary objective mentioned in Section \ref{ch:Introduction:sec:Conceptual Formulation} is performed.
This leads to the design part, which takes the use cases collected in Chapter \ref{ch:Requirements Engineering} and illustrates them in the form of flowcharts that form the basis of the implementation. 
On the basis of the outcomes of the previous analysis and design phase, a description of the translation of the selected processes for implementation in software components within the application follows. This document provides a comprehensive description of the entities, actors, and related systems involved.
Finally, the results are evaluated and validated according to the scenario created, concluding this paper with a summary in combination with impulses for further development and integration of other aspects mentioned in this work.