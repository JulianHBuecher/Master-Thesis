%% LaTeX2e class for student theses
%% thesis.tex
%%
%% Karlsruhe University of Applied Sciences
%% Faculty of  Computer Science and Business Information Systems
%%
%%
%% Version 0.2, 2017-11-15
%%
%% --------------------------------------------------------
%% | Derived from sdqthesis by Erik Burger burger@kit.edu |
%% --------------------------------------------------------

%% Available languages: english,ngerman
%% Available modes: draft,final (see README)
\documentclass[english,final]{thesis}

% Use space between paragraphs
%\KOMAoption{parskip}{half+}

%% ---------------------------------
%% | Information about the thesis  |
%% ---------------------------------

%% Name of the author
\author{Julian Bücher}

%% Title (and possibly subtitle) of the thesis
\title{Reservation process for charging infrastructure management in the context of e-mobility}

%% Type of the thesis
\thesistype{Master Thesis}

%% Change the institute or subject area here, ``VSYS'' is default
% \myinstitute{Institute for \dots}

%% You can put a logo in the ``logos'' directory and include it here
% \grouplogo{myfile}

\reviewerone{Prof. Dr.-Ing. Zoltán Nochta}
%% reviewer two (can be omitted)
\reviewertwo{Prof. Dr. rer. nat. Heiko Körner}

%% The advisor is usually extern (can be omitted)
% \advisorone{Dipl.-Inform. C}
% The second advisor (can be omitted)
% \advisortwo{Dipl.-Inform. D}


%% Please enter the start end end time of your thesis
\editingtime{17. March 2023}{16. September 2023}

\settitle

%% --------------------------------
%% | Settings for word separation |
%% --------------------------------

%% Describe separation hints here.
%% For more details, see
%% http://en.wikibooks.org/wiki/LaTeX/Text_Formatting#Hyphenation
\hyphenation{
% me-ta-mo-del
}

%% --------------------------------
%% | Bibliography                 |
%% --------------------------------

%% Use biber instead of BibTeX, see README
\usepackage[citestyle=numeric,style=numeric,backend=biber]{biblatex}
\addbibresource{thesis.bib}

%% --------------------------------
%% | Listings                     |
%% --------------------------------

%% Basic config
\lstset{
  frame=tb,
  aboveskip=3mm,
  belowskip=3mm,
  showstringspaces=false,
  columns=flexible,
  basicstyle={\footnotesize\ttfamily},
  numbers=left,
  numberstyle=\tiny\color{gray},
  keywordstyle=\color{blue},
  commentstyle=\color{dkgreen},
  stringstyle=\color{mauve},
  breaklines=true,
  breakatwhitespace=true,
  tabsize=3,
  showlines=true
}

%% Add your non-supported language here

\definecolor{lightgray}{rgb}{.9,.9,.9}
\definecolor{darkgray}{rgb}{.4,.4,.4}
\definecolor{purple}{rgb}{0.65, 0.12, 0.82}
\definecolor{blueprimary}{HTML}{1976D2}

\lstdefinelanguage{TypeScript}{
  keywords={typeof, new, true, false, catch, function, return, null, catch, switch, var, if, in, while, do, else, case, break},
  keywordstyle=\color{blue}\bfseries,
  ndkeywords={class, export, boolean, throw, implements, import, this},
  ndkeywordstyle=\color{darkgray}\bfseries,
  identifierstyle=\color{black},
  sensitive=false,
  comment=[l]{//},
  morecomment=[s]{/*}{*/},
  commentstyle=\color{purple}\ttfamily,
  stringstyle=\color{red}\ttfamily,
  morestring=[b]',
  morestring=[b]"
}

%% --------------------------------
%% | Glossaries                   |
%% --------------------------------

%% Use glossaries (optional)
\makeglossaries
%% Load glossary definitions from file
\loadglsentries{glossentries.tex}

%% ====================================
%% ====================================
%% ||                                ||
%% || Beginning of the main document ||
%% ||                                ||
%% ====================================
%% ====================================
\begin{document}

%% Set PDF metadata
\setpdf

%% Set the title
\maketitle

%% The Preamble begins here
\frontmatter

%% LaTeX2e class for student theses: Declaration of independent work
%% sections/declaration.tex
%%
%% Karlsruhe University of Applied Sciences
%% Faculty of  Computer Science and Business Information Systems
%%
%% --------------------------------------------------------
%% | Derived from sdqthesis by Erik Burger burger@kit.edu |
%% --------------------------------------------------------


\thispagestyle{empty}
\null\vfill
\noindent\hbox to \textwidth{\hrulefill}
\iflanguage{english}{I declare that I have developed and written the enclosed
thesis completely by myself, and have not used sources or means without
declaration in the text.}%
{Ich versichere wahrheitsgemäß, die Arbeit
selbstständig angefertigt, alle benutzten Hilfsmittel vollständig und genau
angegeben und alles kenntlich gemacht zu haben, was aus Arbeiten anderer
unverändert oder mit Änderungen entnommen wurde.}


%% ---------------------------------------------
%% | Replace PLACE and DATE with actual values |
%% ---------------------------------------------
\textbf{PLACE, DATE}
\todo{Please replace with actual values}
\vspace{1.5cm}

\dotfill\hspace*{8.0cm}\\
\hspace*{2cm}(\theauthor)
\cleardoublepage


\setcounter{page}{1}
\pagenumbering{roman}

%% ----------------
%% |   Abstract   |
%% ----------------

%% For theses written in English, an abstract both in English
%% and German is mandatory.
%%
%% For theses written in German, a German abstract is sufficient.
%%
%% The text is included from the following files:
%% - sections/abstract

\includeabstract

%% ------------------------
%% |   Table of Contents  |
%% ------------------------
\tableofcontents

\listoffigures
\listoftables

%% ------------------------
%% | Glossary/Acronyms    |
%% ------------------------

% print glossary (optional)
\printglossary
% print acronyms (optional)
\printglossary[type=\acronymtype]

%% -----------------
%% |   Main part   |
%% -----------------

\mainmatter
% Introduction
%% LaTeX2e class for student theses
%% sections/introduction.tex
%%
%% Karlsruhe University of Applied Sciences
%% Faculty of  Computer Science and Business Information Systems
%%
%% --------------------------------------------------------
%% | Derived from sdqthesis by Erik Burger burger@kit.edu |
%% --------------------------------------------------------

\chapter{Introduction}
\label{ch:Introduction}

Regarding the fact of a further decreasing amount of fossil fuels available for the production of products like gasoline, or the resulting consequences like global warming by burning these energy carriers. These are only two arguments in the debate about fostering the development of a more sustainable behavior in different areas of daily life \cite{kathiresh_e-mobility_2022}.
Beside switching from coal or oil as a viable source for energy generation to renewable energies, or the transition from cars powered by \acrfull{ice} to a more environmental friendly alternative, is an essential step reducing the amount of \acrfull{co2} in the atmosphere and decreasing the elevating speed of the climate crisis.
As part of the 'electrification of energy conversion systems in mobile platforms' \cite[165226]{adib_e-mobility_2019} in the context of \acrfull{emobility}, the industry and users still having several challenges to face.
The most outstanding ones are the existing shortcomings in key technologies like batteries or power trains. In comparison to their current fossil alternatives, they lack in several aspects like efficiency or durability, which results in shorter operating ranges regarding to the limited capacities. In combination with the shortage of charging possibilities in sparsely populated areas or cities, the current shortcomings could not be countered effectively.
Other negative site effects like the so called 'range anxiety' \cite{rauh_understanding_2015} experienced by most \acrfull{evu} in rural areas without a widespread charging infrastructure, affected the acceptance and changeover from cars with \acrshort{ice} to \acrfull{hev} or \acrfull{fev} additionally. 
Furthermore, the waste produced by the production or disposal of \acrshort{ev} is a critical point. Especially the disposal of lithium-ion batteries lead to deleterious effects on the surrounding environment \cite{xu_generation_2017}.
Nevertheless, the long term benefits of \acrshort{ev} and their impact in reducing \acrshort{co2} and the utilization of fossil fuels are unmistakable.
Because of this, several institutions and governments are anxious to support the expansion of \acrshort{emobility} and try to counter the negative site effects mentioned above.
For example subsidies and other inducements provided by the manufacturers as well as the state itself are countermeasures to promote \acrshortpl{ev} as a future transportation mean.

\section{Conceptual Formulation}
\label{ch:Introduction:sec:Conceptual Formulation}

Due to the need of a system to administrate and monitor the required infrastructure, different approaches of managing \acrshortpl{cs} and according connectors have been established. Therefore, various software companies partnered with manufacturers of charging stations or institutions for standardization to establish a common denominator for information exchange.
Their work resulted in standards like the \acrfull{ocpp}, \acrfull{ocpi} or ISO 15118, which provide a decent feature set for interaction between the \acrshortpl{evu}, the \acrshortpl{ev} and the corresponding \acrshortpl{cs}. Despite the extensive scope of processes covered, the mentioned standards are not considered as complete and require additional refinements and extensions for certain situations.
Therefore, nearly every year new major versions with minor fixes and improvements in form of new features is released. 
Considering the evaluation of specific feature gaps, especially management or administration processes, ensuring a fair and well-regulated use of the single charging possibilities, is a necessary functionality most implementations are missing nowadays. Addressing this particular use case, this thesis describes a process for reservations of charging points or connectors for a specified time range in advance. Beside providing the \acrshortpl{evu} a guarantee for charging, this feature set should enable a fine grained administration of \acrshortpl{cs} for owners of semi-public or private parking lots and offer an alternative to the first-come-first-serve principle currently used.

\section{SAP SE}
\label{ch:Introduction:sec:SAP SE}

This work was part of a master's thesis at SAP SE, which is a multinational software corporation based in Walldorf, Germany. It is one of the world's leading enterprise software companies and is renowned for its innovative solutions that help businesses manage their operations effectively. It was founded in 1972 by Dietmar Hopp, Hasso Plattner, Claus Wellenreuther, Hans-Werner Hector, and Klaus Tschira \cite{noauthor_inventing_nodate}. 
The company's goal is to develop standard application software for real-time business data processing and improve the way businesses managed their operations. Beside SAP S/4HANA and SAP ERP (Enterprise Resource Planning), SAP offers a wide range of enterprise software products and service and catering to various business needs and industries.
Beside manufacturing industries, the finance and healthcare sector, as well as, retail, utilities, and public sector organizations SAP engages in the development of emerging technologies such as artificial intelligence, machine learning, Internet of Things (IoT), or solutions in the \Gls{emobility} sector.
This includes software for the management of existing \acrshortpl{cs} within the company or applications for supporting the \acrshortpl{evu} by a more comfortable way of finding free charging opportunities. As part of this approach driving the expansion of \Gls{emobility} in the enterprise world, this thesis provides additional concepts for the integration in existing software applications and example workflows for advanced management of the charging infrastructure. 

\newpage

\section{Document Structure}
\label{ch:Introduction:Document Structure}

As main focus of this work, the following chapters describe a potential solution for conceptualization and implementation of a reservation system based on the existing SAP open source solution \textit{Open e-Mobility}. 
Starting with the introduction of the necessary information for contextualization, the conceptional part afterwards gathered the required use cases for the design of the corresponding processes. 
Followed by a structured timeline describing the approaches for the theoretical part as well as the implementation part separated into single steps. 
Next, an evaluation of existing work covering similar approaches for handling the main task mentioned in \ref{ch:Introduction:sec:Conceptual Formulation}, takes place.
This leads to the design part covering the use cases gathered in chapter \ref{ch:Requirements Engineering} and illustrate them in form of flowchart diagrams, which are basis of the implementation. 
Based on the results of the preceding analysis and the designing phase, a documentation describing the translation of the selected processes for implementation in software components inside the application follows. It contains a detailed explanation of the involved entities, actors and the corresponding systems.
Afterwards an evaluation and validation of the results based on the created scenario is appended. 
The closing of this paper is composed of a conclusion in combination with impulses for further development and integration of other aspects touched by this work.

% Main part
%% LaTeX2e class for student theses
%% sections/main/1_fundamentals.tex
%%
%% Karlsruhe University of Applied Sciences
%% Faculty of Computer Science and Business Information Systems
%%
%% --------------------------------------------------------
%% | Derived from sdqthesis by Erik Burger burger@kit.edu |
%% --------------------------------------------------------


\chapter{Fundamentals}
\label{ch:Fundamentals}

The subsequent chapter provides a fundamental understanding of the concepts and general conditions, that apply in the sector of \acrshort{emobility}, and therefore are omnipresent in most of the described scenarios later on.
First of all, Section \ref{ch:Fundamentals:sec:Electric Mobility} gives a general review of \acrshort{emobility}, like the involved actors and their systems. Furthermore, existing standards for depicting the communication between these actors and systems and a brief introduction to smart charging and its variations follow.
Additionally, the subsequent Section \ref{ch:Fundamentals:sec:Reservation Systems} illustrates the basic functionality of a reservation system itself, its use cases, and a differentiation between common types of reservation systems.
Lastly, the Section \ref{ch:Fundamentals:sec:Data Exchange} lists important communication protocols and data exchange formats used in the respective standards described in \ref{ch:Fundamentals:sec:Electric Mobility:ssec:Relevant Standards}.

\section{Electric Mobility}
\label{ch:Fundamentals:sec:Electric Mobility}

The term \acrshort{emobility} generally refers to the use of \acrfull{ev} and other electric--powered transportation options as an alternative to conventional vehicles that are powered by an \acrfull{ice}. 
As a crucial element of the worldwide initiative to counter climate change and achieve sustainable transport solutions, it aims to reduce greenhouse gas emissions, lower dependence on fossil fuels and mitigate the environmental impact of transportation \cite{kathiresh_e-mobility_2022}.
To achieve this objective, \acrshort{emobility} depends on the expansion of \acrshortpl{ev} as a common means of transport and a denser cluster of \acrshortpl{cs}.
Apart from the focus on vehicles primarily addressing personal transportation, this term could also be applied in the context of freight transport, which demands much higher capacities in terms of the power required by such vehicles.
Within this research and the following chapters, only \acrshortpl{bev} used for personal transportation in the form of cars or smaller vans are considered as the respective audience.

\subsection{Electric Vehicles}
\label{ch:Fundamentals:sec:Electric Mobility:Electric Vehicles}

As already mentioned in Section \ref{ch:Fundamentals:sec:Electric Mobility}, \acrshortpl{ev} are a fundamental cornerstone of \acrshort{emobility}. Primarily, an \acrshort{ev} refers to a car that is powered by only one or more electric engines drawing energy from onboard batteries or similar energy sources. 
Besides \acrfullpl{bev}, which are purely powered by electricity, further invariants like \acrfullpl{phev} or \acrfullpl{hev} exist. In comparison to the aforementioned \acrshortpl{bev}, \acrshortpl{phev} combine the power of an electric motor with the reliance offered by an internal combustion engine.
This allows the driver to cover shorter distances solely relying on the internal battery, and use the petrol engine during longer trips or when charging is unavailable. Moreover, \acrshortpl{phev} have the capability of recharging the battery while on the move.
Therefore, brake resistance is employed to convert the power generated into electricity, feeding it directly back into the battery.
\acrshortpl{hev} closely resemble \acrshortpl{phev} as they have access to both power sources. However, unlike a \acrshort{phev}, \acrshortpl{hev} do not have the possibility to be plugged in for charging \cite{kathiresh_e-mobility_2022}.

\begin{figure}[h]
    \centering
    \includegraphics[width=0.8\textwidth]{resources/images/main/1_fundamentals/ElectricVehicleTypes.png}
    \caption{Classification approach for different \acrshort{ev} types based on \cite{acharige_review_2023}.}
    \label{fig:ev-classification}
\end{figure}

\subsection{Charging Infrastructure}
\label{ch:Fundamentals:sec:Electric Mobility:ssec:Charging Infrastructure}

To facilitate the growing number of \acrshortpl{ev}, a crucial aspect is the establishment of a broad charging infrastructure comprising multiple \acrshortpl{cs}. As highlighted in \cite{gnann_fast_2018}, the requirements for charging infrastructure are highly variable, depending on the battery capacity and charging rate, which is predicted to increase in the future.
In order to satisfy the various power requirements and to cover the wide range of vehicles and used batteries, a heterogeneous range of \acrshortpl{cs} exists.
From stations providing so--called slow charging with a power supply up to 22 kW, more advanced technologies allow fast charging \acrshortpl{cs} for power requirements above this certain range. \\
\noindent For a better understanding of the different scenarios in which a \acrshort{cs} could be used for charging, this work divides the infrastructure and the corresponding stations into three different groups. 
These groups, classified according to accessibility level for the \acrshortpl{evu}, are presented in Table \ref{tab:cs-accessibility-levels} alongside their user restrictions, the associated locations, and most commonly used charger levels:

\begingroup
\setlength{\tabcolsep}{10pt} % Default value: 6pt
\renewcommand{\arraystretch}{1.5} % Default value: 1
\begin{table}[h]
\centering
\caption{Differentiation of accessibility in case of charging opportunities based on \cite{kathiresh_e-mobility_2022},\cite[18-19]{linnemann_elektromobilitat_2020}.}
    \begin{tabular}{c|c|m{5.5cm}|c}
    Accessibility & Restrictions & Location & Charging Level \\ \hline
    Public & No & fleets, highway, distribution centers & 3 \\
    Semi--Public & Yes & workplace, hotels & 2 \\
    Private & Yes & private households & 1 
    \end{tabular}
\label{tab:cs-accessibility-levels}
\end{table}
\endgroup

\newpage

\noindent Alongside the separation according to the accessibility level, a differentiation based on classes of \acrfullpl{cs} is possible. For a detailed illustration of the dependencies and relationships between the different classes, see Figure~\ref{fig:charging-station-classification} below. 

\begin{figure}[h]
\centering
\includegraphics[width=0.8\textwidth]{resources/images/main/1_fundamentals/ChargingStationClassification.png}
\caption{Classification for different types of \acrshortpl{cs} based on \cite{afshar_literature_2020}.}
\label{fig:charging-station-classification}
\end{figure}

\noindent To provide a more detailed view on the infrastructure and the hardware used for charging, also referred to as \acrfull{evse}, all components and systems necessary to supply electric power from the grid to the battery of an \acrfull{ev} are explained afterwards.
Beginning with the \textbf{\acrfull{cs}} itself, as physical power unit the \acrshort{ev} is connected to. As mentioned above, \acrshortpl{cs} can vary in size and complexity, ranging from small wall--mounted units designed for home use to more extensive public \acrshortpl{cs} found in parking lots, shopping centers or along highways. 
To establish a connection between grid, \acrshort{cs} and car, a \textbf{Connector or Charging Cable} is needed. It contains plugs on both ends, which are connecting to the vehicle's charging port and to the \acrshortpl{cs} outlet. 
As an integral part of the \acrshort{evse}, the \textbf{\acrshort{pms}} regulates the flow of electricity from the grid to the vehicle's battery. It ensures safe and efficient charging, managing the power level based on the vehicle's battery capacity and the grid's capacity.
Allowing the communication between the \acrshort{ev}, the \acrfull{cs}, and the grid, a \textbf{\acrshort{cms}} is used as a backend. It allows data exchange regarding charging status, electricity prices, user authentication, and other relevant information.
In the case of public \acrshortpl{cs}, a payment and authentication system is usually integrated into the \acrshort{evse}. This system verifies the user's identity, authorizes the charging session, and processes payment for the electricity consumed.
On top of the described components, various safety features are integrated to protect users, the vehicle, and the surrounding environment. Including features such as ground fault protection, over--current protection, temperature monitoring, and emergency shut--off capabilities~\cite{littlefuse_designing_2020}.
\noindent To meet the wide range of existing \acrshortpl{ev} and their individual power requirements, different categories of chargers have been established based on the power level provided. This work lists the most notable categories, along with their specifications and typical usage, in Table~\ref{tab:ev-charging-levels}. 

\begingroup
\setlength{\tabcolsep}{10pt} % Default value: 6pt
\renewcommand{\arraystretch}{1.5} % Default value: 1
\begin{table}[h]
    \centering
    \caption{Listing of different \acrshort{ev} charging levels based on \cite{acharige_review_2023}.}
    \begin{tabular}{c|c|m{6.5cm}}
        Charging Level & Charging Power & Use Case \\ \hline
         1 & 1.44 kW -- 1.9 kW & Typically the slowest charging option and suitable for overnight charging at home \\
         2 & 3.1 kW -- 19.2 kW & Provides faster charging compared to Level 1 and is commonly used for home charging setups and public charging stations \\
         3 & 20 kW -- 350 kW & Fast charging operates at a higher voltage, directly charging the vehicle's battery with direct current power. Commonly used in public fast--charging stations.\\
         \acrshort{xfc} & >350 kW & Ultra--fast charging provides fastest charging option. Commonly used in commercial settings. 
    \end{tabular}
    \label{tab:ev-charging-levels}
\end{table}
\endgroup

\newpage

\subsection{Battery Technology}
\label{ch:Fundamentals:sec:Electric Mobility:ssec:Battery Technology}

For storing the required energy, \acrshortpl{ev} rely on their built--in batteries. As key components inside an \acrshort{ev}, they have to handle high energy capacity and high power, within limited weight and space to affordable prices.
Therefore, most manufacturers use Lithium--ion batteries in their cars nowadays. 
Alternatives such as \textit{Lead Acid} and \textit{Nickel--based} batteries usually have lower durability compared to lithium--based batteries due to their shorter lifespan and subsequent inadequate performance in extreme weather or higher discharge rates \cite{acharige_review_2023}.
Generally, the advances in this technology leading to an increase in the range of the \acrshortpl{ev} and a reduction in the charging time, combined with cost effectiveness, are an essential part of the transition from cars using \acrshort{ice} to purely \acrshort{ev}.

\subsection{Relevant Standards}
\label{ch:Fundamentals:sec:Electric Mobility:ssec:Relevant Standards}

To establish a comprehensive interface for information exchange between \acrshortpl{cs}, \acrshortpl{ev}, and their users, various organizations, initiatives and the industry have established several standards for communication between the single actors in this particular environment.
Aside from the \acrfull{ocpp} \cite{noauthor_ocpp_nodate}, a communication protocol, developed by the \acrfull{oca} \cite{noauthor_open_nodate}, other protocols and specifications exist, covering different aspects and scenarios relevant during or before the charging process.
In addition to the previously mentioned \acrshort{ocpp}, further standards relevant to this work are listed and explained below.

\subsubsection{Open Charge Point Protocol}
\label{ch:Fundamentals:sec:Electric Mobility:ssec:Relevant Standards:sssec:OCPP}

The \acrfull{ocpp} outlines an open industry--standard communication protocol, used especially in charging infrastructure for \acrshortpl{ev}. Proposed by the ElaadNL foundation as an open protocol to support the communication between \acrshort{cs} and the related backend services, the ownership was transferred to \acrshort{oca} in 2014 \cite{garofalaki_electric_2022}.
In general, the \acrshort{ocpp} is designed to provide interoperability and seamless integration between different \acrshort{cs} vendors and network operators, ensuring that \acrshort{ev} drivers can charge their vehicles at any compatible \acrshort{cs}.
Therefore, it is maintained by the \acrfull{oca}, a consortium of \acrshort{ev} charging infrastructure stakeholders, as an open protocol and not proprietary to any specific manufacturer or organization. 
In this way, it is ensured that the \acrshort{ocpp} remains a collaborative and evolving standard \cite{noauthor_ocpp_nodate}. \\
From an architectural point of view, the protocol could be described in the form of a client/server architecture. The \acrshort{evse} in this model acts as the client, while the \acrshort{csms} serves as the server. 
The server responds to the client's requests and manages the charging processes accordingly. This request--response model, where the client sends requests to the server, and the server responds with the appropriate information or action enables real--time communication between the \acrfull{cs} and the connected \acrfull{cms}.
As an interface for communication, two different types of protocols are used. On the one side, the WebSocket protocol, described in \ref{ch:Fundamentals:sec:Data Exchange:ssec:WebSocket}, for bidirectional communication, providing a persistent connection between the \acrshort{evse} and the \acrshort{cms} allowing a faster and more efficient data exchange. On the other side, \acrfull{soap}, as described in \ref{ch:Fundamentals:sec:Data Exchange:ssec:SOAP}, is suitable for the implementation of one--way communication between the respective entities.
To differentiate between particular feature sets, the \acrshort{ocpp} is versioned. 
The versions are presented as floating point numbers, which increase with major or minor releases within the nomenclature, such as \acrshort{ocpp} 1.6 and \acrshort{ocpp} 2.0. Representing the latest versions of the protocol, which are widely used in current implementations. \\ \\
\noindent For the communication between \acrshort{cs}, \acrshort{csms} and the \acrshort{evu}, the \acrshort{ocpp} protocol relies on so called operations. Each of these operations describes a set of instructions, which are necessary to fulfill and successfully complete the underlying process.
According to the required operations used in the context of this work, a subset of operations are selected, which are elucidated in the following list. 
The wording and the explanations are based on the standard documentation for the \acrshort{ocpp} version 1.6 \cite{noauthor_ocpp_nodate}:

\begin{description}
    \item[Authorize] Before starting a charging session, the \acrshort{evu} \acrfull{rfid} card or other identification is sent to the central system for authorization. The \acrshort{cms} checks the driver's credentials and responds with an authorization status.
    \item[Start Transaction] After successful authorization, the \acrshort{cms} sends a start transaction request to initiate the charging process. The \acrshort{cs} acknowledges the request, and the charging session begins.
    \item[Status Notification] The \acrshort{cs} may send status notifications to the \acrshort{cms} to update the current state of the \acrshort{cs}, such as \verb|Available|, \verb|Charging|, \verb|Reserved|, or \verb|Faulted|.
    \item[ReserveNow] In case a user needs an available connector on a \acrshort{cs}, he could send a \textit{ReserveNow} request via the \acrshort{cms} to the \acrshort{cs}, which reserves one specific or at least one connector on the station for a specified duration. The according connector in case of successful reservation changes from status \verb|Available| to \verb|Reserved|.
    As a result, only the user assigned to the deposited \acrshort{rfid} card is able to start charging at the specified connector or the superior \acrshort{cs}.
    \item[Cancel Reservation] For canceling the created reservation the user has the ability to manually send a request with the assigned reservation \acrshort{id} to the \acrshort{cms}, to free the connector again. Otherwise, the \acrshort{cs} will notify the \acrshort{cms}, if the reservation reaches its expiry limit.
\end{description}

\subsubsection{Open Charge Point Interface}
\label{ch:Fundamentals:sec:Electric Mobility:ssec:Relevant Standards:sssec:OCPI}

\acrfull{ocpi} is a standardized, open protocol facilitating communication and supporting interoperability between different charging network operators, which provides \acrshort{ev} drivers with access to charging infrastructure from multiple providers. This is achieved by utilizing a harmonized approach. 
Alongside \acrshort{ocpi}, other proposals regarding the development of cross--border \acrshort{ev} travel, also known as \textit{e--roaming}, exist. 
The most prominent ones are \acrfull{oicp}, \acrfull{ochp} and \acrfull{emip} \cite{ferwerda_advancing_2018}. 
In contrast to the \acrshort{ocpi}, they are all developed by proprietary institutions and integrated inside their offerings providing dedicated roaming platforms. 
Thus, no way is provided to allow compatibility, making the \acrshort{ocpi} standard more suitable in the case of openness and interoperability. \\
\noindent In case of available features, \acrshort{ocpi} provides defined endpoints, used for communication between \acrshort{cs}, \acrshortpl{cms}, \acrshortpl{emsp} and \acrshortpl{cpo}. These endpoints include functionalities such as location discovery, charge point data, authorization, charging sessions, and error handling. 
For integrating \acrshort{ocpi} into existing software systems it is heavily based on the paradigm of \acrfull{rest}. This allows communication via standard HTTP methods utilizing \acrfull{json} as a data format for transmitting information. 

\noindent The listing below provides a short overview of available features and functionalities provided by \acrshort{ocpi} \cite{noauthor_open_2021}:

\begin{description}
    \item[Location Discovery] Charging networks can exchange information about available charging locations, providing details such as location coordinates, \acrshort{cs} types, and status.
    \item[Charge Point Data] Access to data on \acrshort{cs}, including details on availability, status, and pricing, is possible through \acrshort{ocpi}.
    \item[Charging Sessions] \acrshort{ocpi} supports real--time information about ongoing charging sessions, including start time, energy consumed, and current charging status.
    \item[Authorization and Authentication] In combination with their respective \acrshort{cpo} accounts or other authentication methods \acrshortpl{evu} are able to authenticate themselves to access charging services.
    \item[Remote Start/Stop Charging Sessions] To initiate charging sessions, \acrshortpl{evu} are capable of remotely starting and stopping their current charging sessions. This allows the driver to initiate charging sessions via their mobile apps or other remote means.
    \item[Tariff Information] Charging networks can share pricing and tariff information, providing transparency to \acrshortpl{evu} about the cost of charging at different locations.
\end{description}

\subsubsection{ISO 15118}
\label{ch:Fundamentals:sec:Electric Mobility:ssec:Relevant Standards:sssec:ISO 15118}

To meet the demands of \acrshortpl{evu} and address the problems of lack of standardization in charging infrastructure interoperability and information exchange between grid \acrshort{evse}, \acrshort{ev} and \acrshort{evu}, Daimler and RWE started a collaboration in September 2009 to enable Smart Charge Communication.
Until its initial release as the standard for Plug--n--Charge connections in June 2014, further progress towards the standardization of charging infrastructure communication was carried out \cite{heinrich_iso_2017}.
Nowadays, this standard consists of multiple parts. From the analysis of the underlying use cases to a description of requirements for the communication over the different \acrshort{osi} layers, it describes \acrshort{poi}--data, the \acrshort{cs} availability, payment, communication standards as well as the \textit{Plug \& Charge} or \textit{e--Roaming} approach \cite{brosi_methode_2019}. 
However, the required level of data granularity or quality is not specified \cite{linnemann_elektromobilitat_2020}. Considering these gaps that still exist in the \acrshort{iso} 15118, it is far from being complete and requires further development and adaptation.

\subsection{Smart Charging}
\label{ch:Fundamentals:sec:Electric Mobility:ssec:Smart Charging}

The concept of smart charging or intelligent charging is a systematic approach that optimizes the charging process of \acrshortpl{ev} to be more efficient, cost--effective, and environmentally friendly using information and communication technologies. 
Therefore, the subsequent calculations consider factors such as electricity demand, grid capacity, availability of renewable energy, and individual user preferences to allow for optimizations in the user's decision--making \cite{deb_smart_2022}.
One of the primary goals of this methodology is to balance the demand on the underlying grid, affected by charging large numbers of \acrshortpl{ev} simultaneously \cite{daina_electric_2017}. Especially during peak hours increased electricity costs or potential blackouts could occur. 
In order to mitigate these situations, smart charging takes the current load of the grid into account and adjusts the possible charging rates based on the given constraints to prevent overloading the grid and optimizing the use of available energy \cite{garcia-villalobos_plug-electric_2014}. 
Compared to the unidirectional charging method originally known as \acrfull{v1g}, the approach of \acrfull{v2g} enables smart charging systems to charge bidirectionally between the \acrshortpl{ev} battery and the grid \cite[199]{kathiresh_e-mobility_2022}. Allowing \acrshortpl{ev} with sufficient battery capacity to return energy to the grid during times of high demand and turning them into mobile energy storage units to enhance grid stability and resiliency.
To offer all these functions, smart charging relies on data connectivity and communication between the single \acrshortpl{cs}, the \acrshortpl{ev}, and the underlying grid, which makes it vulnerable to malign third--party entities as well. On the other side, these factors can contribute to better grid management overall, giving grid operators insight into electricity demand patterns and enabling them to plan grid upgrades accordingly.

\subsection{Smart Grid}
\label{ch:Fundamentals:sec:Electric Mobility:ssec:Smart Grid}

More generally, the terminology of a \acrfull{sg} describes the characteristics of an intelligent system that deals with high energy consumption in order to increase energy reliability and the corresponding costs \cite{sharma_smart_2020,moreno_escobar_comprehensive_2021}.
It consists of multiple parts, like a sufficient infrastructure, applications, and several technologies to manage the energy flow inside the grid.
This includes, for example, monitoring and measuring solutions as important cornerstones, allowing the grid to support an efficient generation and distribution of power. Supplemented by the behavior of the users connected to the \acrshort{sg}, which is a component of the continuously operating optimization procedures.
Based on \cite{moreno_escobar_comprehensive_2021}, researchers have identified 94 potential optimization algorithms thus far, primarily implemented in central orchestration systems and observing the \acrshort{sg} to manage these kinds of distributed networks.
As a result, the smart grid is capable of dynamically and intelligently controlling the flow of power based on predetermined objectives to meet the needs of all nodes in the network. 
Considering the delivery of electricity, compared to existing grids that deliver power in one direction only, smart grids enable bi--directional power delivery.  
Therefore, the smart grid utilizes \acrshort{der}, such as \acrshortpl{ev} or renewable energy sources, located at consumer homes, to absorb surplus power when not in use.
This approach is very similar to the \acrshort{v2g} functionality of smart charging described in \ref{ch:Fundamentals:sec:Electric Mobility:ssec:Smart Charging} and could be combined with it.

\section{Reservation Systems}
\label{ch:Fundamentals:sec:Reservation Systems}

Originally, current reservation systems were developed as \acrfull{crs} to facilitate collaboration between airlines and technology companies \cite{xiang_evolution_2020}. The original objective of the system to be a tool to book resources without the need for paper or other management systems has obviously evolved over time.
This also applies to the features and abilities of these systems.
Started as a centrally managed system with restricted access to call center operators, which administered the available resources for their corresponding airlines or hotel chains, the further development of the world wide web in 1989 led to a transformation away from the \acrshort{crs} to online booking systems, also known as web--based \acrfullpl{ibe}, directly available to the customer via their personal computers at home.
In general, a reservation system could be defined as a software application or platform that facilitates the process of booking and securing limited services, resources, or accommodations in advance, based on the definition of Xiang et al. \cite[p.~2]{xiang_evolution_2020}. 
As mentioned previously, besides its original application in the travel or hospitality industry, it is also a common tool for the administration of the transportation or entertainment industry. With the goal of simplifying reservation procedures for customers and minimizing administrative tasks, while increasing user satisfaction.
Based on the system architecture and the technologies employed, reservation systems could vary depending on the functionalities required. When utilized in a web--based application, such as an online booking system, it is essential to consider different constraints and use cases, compared to a local system within one institution. 
To abstract from the requirements necessary for particular deployment scenarios, the following section outlines a set of fundamental characteristics and functionalities that a reservation system should typically encompass. While the chosen list of features may not include all the use cases required for a specific system, it should furnish a brief overview for general purposes.
For reference and guidance in the case of the functionality of the reservation system, the existing research and literature on hotel reservation systems such as \cite{delizo_online_2013,bemile_online_nodate} as well as the common functionalities found in current booking systems are used.
\textbf{Booking and Scheduling} could be described as the main purpose of a reservation system. By using this feature users can view available dates, times, or slots and make reservations for specific services, activities, or resources. The system ensures that conflicting bookings do not occur.
To provide an overview, which represents the availability of the aforementioned resources, the system requires \textbf{Real--Time Availability}. Therefore, users can see immediately if the desired service or resource is available for the chosen date and time.
For the purpose of personalizing the user experience, retaining reservation history, and enabling loyalty rewards, the common reservation application provides a \textbf{User Registration and Authentication} feature to allow the user to create an account or log into restricted areas of the system to book a reservation.
Additionally, managing and monitoring bookings, the option to securely pay for allocated properties using a credit card or digital wallet via \textbf{Online Payments} often is part of the service offering.
To acknowledge successful reservations or notify users of any changes to their bookings, a \textbf{Confirmation and Notifications} should be embedded in the application. This feature not only acts as a confirmation of the booking, but also evidence of the completed transaction and may contain information such as booking \acrshort{id}, date, time, and location.
Regarding the need for modification or cancellation of the reservation, the system should provide \textbf{Cancellation and Modification} features, enabling the user to cancel or modify their reservations within a specified time frame. Additionally, cancellation policies eventually exist for penalties punishing late cancellations.
Moreover, apart from the management of the reservations made through the system, it should provide a \textbf{Inventory Management} for the owners of the resources and help them manage their possessions.
Other features such as \textbf{Reporting and Analytics} or the \textbf{Integration with Other Systems}, which allow the generation of reports on booking trends, revenue, and occupancy rates, are advanced functionalities typically not visible to the end--user. Mostly offered as an administrative service extending the basic use of the system, reporting and third--party integration of other systems, such as customer relationship management (CRM), is not part of every software suite. 
This also applies to the integration of payment gateways.

\section{Data Exchange}
\label{ch:Fundamentals:sec:Data Exchange}

After the presentation of the various services, their corresponding communication channels, and the information provided within Section \ref{ch:Fundamentals:sec:Electric Mobility:ssec:Relevant Standards}, the subsequent section deals in more detail with the technical concepts that underlie both the protocols and the application.
In addition to a brief overview of \acrfull{rest} in \ref{ch:Fundamentals:sec:Data Exchange:ssec:REST}, a quick summary of the technologies \acrfull{soap} in \ref{ch:Fundamentals:sec:Data Exchange:ssec:SOAP}, \acrfull{odata} in \ref{ch:Fundamentals:sec:Data Exchange:ssec:OData} and WebSockets in \ref{ch:Fundamentals:sec:Data Exchange:ssec:WebSocket} follows. 

\subsection{REST}
\label{ch:Fundamentals:sec:Data Exchange:ssec:REST}

Introduced by Roy Fielding as an architectural style \cite{patni_pro_2017} to design distributed systems, \acrfull{rest} provides concepts and a set of constraints that define resources accessed via a common interface by \acrshort{http} methods specified in the standard protocol description.
Stated as the general set of \acrshort{http} methods used to query a \acrshort{rest} \acrfullpl{api}, the methods \verb|GET|, \verb|POST|, \verb|PUT| and \verb|DELETE| are the most conventional examples in the context of web development. 
Representing a set of standardized operations, also known as \acrshort{crud} operations, similar to the functionalities utilized in the context of relational and non--relational databases, these methods assume the activation of certain functionalities within the respective targets.
The verbs used to name the operations carry significant meaning, indicating the associated functionality and the expected outcome of the call.
Aside from the mere retrieval of data, which is usually performed by the use of \verb|GET|, the creation and update of an entity involves either \verb|POST| or \verb|PUT|.
Intentionally, while awaiting a specific result from performing a read operation, creation methods may or may not return the resulting object. Both possibilities align with the standard and result in subtly different behaviors due to varying design decisions, such as prioritizing reducing network bandwidth versus usability.
To determine the appropriate targets for the client, \acrshort{rest} defines the concept resources, which allows a resource identifier to be mapped in the form of a \acrshort{uri} or a \acrshort{urn}. Following this pattern, a legitimate resource representing a car is presumably indicated by the  \verb|http://rest.service.com/cars|. 
Sending a \acrshort{http} request with the \verb|GET| method to this resource should result in a list of cars provided by the service. As a valid format for representing data, both \acrshort{json} and \acrshort{html} documents are permissible. Even images could be utilized.
Further data elements defined by \acrshort{rest} are resource metadata, representation metadata, and control data, which are only mentioned here for completeness. 
In addition to these core components, \acrshort{rest} provides numerous options for application and system design. However, they are not relevant to this research and are therefore not described in detail here.

\subsection{SOAP}
\label{ch:Fundamentals:sec:Data Exchange:ssec:SOAP}

As a counterpart to the \acrshort{rest} protocol described in the previous Section \ref{ch:Fundamentals:sec:Data Exchange:ssec:REST}, \acrshort{soap} was created by Dave Winer and others in collaboration with Microsoft \cite{patni_pro_2017} to addresses the demands of enterprise software.
The fundamental idea of \acrshort{soap} is to provide data as services, tagged with names consisting of both verbs and nouns. For instance, the service named \textit{getCar} is exclusively for providing information about cars and is not intended for any other objectives.
In contrast to \acrshort{rest}, this ensures better communication of the purpose for which the method is intended, as well as the necessary data that this particular function relies on to fulfill its purpose. All of this information about the handled objects and parameters is communicated to the client prior to the actual interaction and described as part of a \acrfull{wsdl} document, which is publicly available from the \acrshort{url} of the particular services implementing this protocol.
This creates a so--called contract between a server and a client, providing security and authorization as well as more direct access to objects available on the server. As a result, these components are much more tightly coupled.
Therefore, \acrshort{soap}, as outlined in \cite[4]{patni_pro_2017}, is typically recommended for scenarios involving use cases considering transactional operations or in case the environment is dependent on this protocol.

\subsection{OData}
\label{ch:Fundamentals:sec:Data Exchange:ssec:OData}

Another option for implementing web--based services besides utilizing only the \acrshort{rest} architectural style is the \acrshort{odata} protocol, which aims to build on top of the latter to provide both a higher level of consistency in terms of breaking changes and to increase interoperability between the communication partners \cite{noauthor_documentation_nodate}.
To accomplish these goals, \acrshort{odata} services, similar to services implementing \acrshort{soap}, supply a metadata definition using \acrfull{csdl} as a metadata description language ensuring a machine--readable form of the service functionalities it provides.
In the case of generic clients with no knowledge of the implemented logic of the services, these documents exchanged as part of a preflight request between server and requestor, give the clients the ability to interact in a formalized way.
Alongside these outstanding features, the \acrshort{odata} standard offers additional concepts such as \textit{namespaces} or specific \acrshort{url} conventions to cover much more advanced scenarios. 
Nevertheless, these aspects are beyond the scope of this informative section and are not covered in this brief introduction.

\subsection{WebSocket}
\label{ch:Fundamentals:sec:Data Exchange:ssec:WebSocket}

Based on the RFC 6455 standard \cite{melnikov_websocket_2011}, the WebSocket protocol facilitates two--way communication between a client and a remote host. Formerly, clients on the web who needed to establish bidirectional communication with their server had to utilize \acrshort{http} calls to request updates from their correspondent and send notifications in response.
Resulting in numerous obstacles for the server and its client, including the quantity of open \acrfull{tcp} connections and the overhead generated by the bloating headers appended to each request.
Trying to solve these problems, the WebSocket protocol suggests using only one \acrshort{tcp} connection for traffic in both directions.
In order to create this information channel, Fette and Melnikov's proposal allows taking advantage of existing infrastructure, which lets the users replace existing technologies using \acrshort{http} as the transport layer within their applications.
To initiate the data exchange process, this communication approach relies on establishing trust through exchanging preliminary handshakes and utilizing the origin--based security model, which is commonly implemented in current browser applications.
Once the trust between server and client is established, the protocol allows the applications to communicate utilizing basic message framing layered over \acrshort{tcp} stack. This grants web--based services the ability to construct efficient information exchange in two directions, without the need to open multiple \acrshort{http} connections.
Apart from the possibility of using the WebSocket protocol over unencrypted \acrshort{http} channels, it can be utilized in encrypted traffic via \acrfull{https} on port 443 as well.

%% LaTeX2e class for student theses
%% sections/main/2_requirements_engineering.tex
%%
%% Karlsruhe University of Applied Sciences
%% Faculty of  Computer Science and Business Information Systems
%%
%% --------------------------------------------------------
%% | Derived from sdqthesis by Erik Burger burger@kit.edu |
%% --------------------------------------------------------

\chapter{Requirements Engineering}
\label{ch:Requirements Engineering}

Main focus of this chapter is to describe the requirements a reservation system for charging infrastructure has to implement.
Therefore, this work uses a fictional scenario to paraphrase and describe the underlying problem. Furthermore, a comprehensive representation of significant edge cases the system design has to consider and highlight potential problems is included.
Adapted from the requirements, use cases and goals are defined, giving an overview of the central tasks of this thesis.

\section{Scenario}
\label{ch:Requirements Engineering:sec:Scenario}

Beginning with the capturing of basic points essential for the outline of the scenario, representing the problem space.
The scenario includes, beside the location the subsequent setting takes place, a group of personas describing fictional people representing actors, who interact with the system itself.
Each of these actors have, based on their current life situation, other requirements regarding the usage of the system, which results in different expectations regarding the fulfillment of their goals based on the interaction with the system.
Formalizing this expectations and transform them in a more general way, they could be derived as use cases separated by system actors for the implementation afterwards.

\noindent To cover a broad range of possible users according to age, social backgrounds and the ease of handling obstacles generated by new technologies, the scenario takes place in the context of the institution \acrfull{hka} located in the city of Karlsruhe.
As partner of the Green4Ever project \cite{noauthor_hka_nodate}, the \acrshort{hka} covers several aspects, which could be used for designing a sophisticated scenario. Beside a broad range of users, ranging from students to the regular staff, it consists of multiple campus sites distributed all over the city.
This allows members of the institution to travel between the single sites and generate persistent demand for charging possibilities during the whole day.
Relevant for the description of the single actors, this scenario includes the sites \textbf{main campus}, \textbf{campus Amalienstraße}, also known as \textit{field office Amalienstraße}, and the \textbf{\acrfull{ltc}} with the amount of applicable \acrshortpl{cs}, which are listed in \ref{tab:campus-sites}.

\begingroup
\setlength{\tabcolsep}{10pt} % Default value: 6pt
\renewcommand{\arraystretch}{1.5} % Default value: 1
\begin{table}[!ht]
    \centering
    \caption{Campus sites and parking lots (site areas) in the \acrshort{hka} organization with the according number of available \acrshortpl{cs}}
    \begin{tabular}{c|c|c}
        Site & Site Area & Number of \acrshortpl{cs} \\
        \hline
        Linder Technologie Campus & Private Parking Lot \acrshort{ltc} & 1 \\
        Linder Technologie Campus & Parking Lot \acrshort{ltc} & 3 \\
        Field Office Amalienstraße & Parking Lot Amalienstraße & 3 \\
        Main Campus & Parking Lot Steinbeishaus & 3 \\
    \end{tabular}
    \label{tab:campus-sites}
\end{table}
\endgroup

\noindent Furthermore, this work assumes that the common user would prefer to gap the distances between the single campuses by public transportation or with their own cars.
As mentioned before, each of these campuses has its own parking lots equipped with a limited amount of \acrshortpl{cs} for students and the staff working at this site. Corresponding to the accessibility level declared in \ref{tab:cs-accessibility-levels}, the parking possibilities includes public and semi-public charging opportunities.
Public parking lots and the corresponding \acrshortpl{cs} are available for students and staff members as well. In the case of semi-public parking lots, the access is only limited to visitors or staff exclusively. \\
Regarding the possibilities to configure \acrshortpl{cs}, all public stations are enabled for reservations. To prevent unnecessary complexity, this scenario does not include \acrshortpl{cs} without support for the reservation feature.
For administration and management of these facilities and the restricted charging resources, a software solution called 'Open e-Mobility' \cite{noauthor_open_2023} is used, which includes a mobile application for the \acrshortpl{evu} and a web-based application for browser access, mostly used by administrative departments of the \acrshort{hka}. 
On behalf of these applications, the users have the possibility, according to their roles in the system listed in \ref{tab:system-role-collection}, to start or stop current charging sessions remotely or check for occupied \acrshortpl{cs} on the available sites and site areas.
Using privileged access via an administrative account, the user are able to deactivate or block connectors of specific \acrshortpl{cs}, to make them inaccessible until reactivation or a rebooting process.
By now, the administrator or the common user does not have the opportunity to reserve a available \acrshort{cs} connector in advance. Rather for themselves or on behalf of a specific user. This includes reservations for visitors arriving later or during times with high load as well.
As an implication of this missing feature, a 'first-come-first-serve' mentality among the \acrshortpl{evu} established, which led to decreased usage of \acrshortpl{ev} as transportation mean and an increasing competitive behaviour among the drivers.

\section{Stakeholders}
\label{ch:Requirements Engineering:sec:Stakeholders}

According to the scenario elaborated in the section \ref{ch:Requirements Engineering:sec:Scenario} above, the following groups of stakeholders could be identified. 
In case of the staff, defined in \ref{ch:Requirements Engineering:sec:Stakeholders:ssec:Staff}, an additional disjunction into two subsidiary clusters is possible. 
The description below should provide an overview of the daily challenges members of these classes of people have to face regarding the established implications defined by the scenario and which main objectives they have. 

\subsection{Student}
\label{ch:Requirements Engineering:sec:Stakeholders:ssec:Student}

Students represent the largest category of actors within the scenario and describe people studying at the \acrshort{hka}. Primarily, students spent most of their time on the \textbf{main campus} site, because the main part of lectures offered by the different faculties and the attendant exercises are located there.
In certain cases, some lecturers offer subjects at other campus positions, where doctoral research projects or research in general is outsourced. This involves several students each semester, commuting between the sites and generating additional demand on the existing charging infrastructure.
Due to the fact, that the amount of students with \acrshortpl{ev} outnumber the capacity of \acrshortpl{cs} at the different sites of the \acrshort{hka}, a student typically concerns about the occupation rate of the public available \acrshortpl{cs}.
In consideration of planning their charging sessions, the students want an overview of available \acrshortpl{cs} at their campus site and want to reserve a public \acrshort{cs}, if they are available for reservation.
Therefore, they are mostly relying on the mobile application offered by the institution, which serves as main entry point for them to interact with the \acrshort{csms}. 

\subsection{Staff}
\label{ch:Requirements Engineering:sec:Stakeholders:ssec:Staff}

The staff of the \acrshort{hka} describes a group of people working at the institution, including roles like lecturers, cleaners or librarians for example.
Based on their office location, they are typically assigned to one specific or several campus sites. Implicating the dedicated usage of the \acrshortpl{cs} for charging their \acrshort{ev}, available at their current working place.
In contrast to students, they have the permission to access the \acrshortpl{cs} on the semi-public parking lots as well as the ones located on the public parking spaces. 
Due to the new home office regulations, staff members have the choice to work from home several days a week, which should relax the situation in regard to the high demand of charging infrastructure.
However, on office days staff members prefer to know the occupation rate of the \acrshortpl{cs} on the public or semi-public spaces and reserve a spot for a guaranteed charging possibility for their \acrshort{ev}.
Referring to the fact, that they spent most part of their daily working hours on the computer rather than their mobile, staff members utilize the web application as primary way of managing their charging sessions.
The mobile application, the majority of the users rely on, is only used in certain cases.

\subsubsection{Maintenance Personal}
\label{ch:Requirements Engineering:sec:Stakeholders:ssec:Staff:sssec:Maintenance Personal}

As dedicated subset of the regular staff working at the \acrshort{hka}, the maintenance personal consists of caretakers in the first place. Their regular occupation insists of responsibility for the maintenance of the physical infrastructure as well as service offerings regarding the charging stations located on the different campuses.
For work-related trips between the single sites, the maintenance team takes advantage of a \acrshort{fev} with a dedicated \acrshort{cs} exclusively accessible by this vehicle and the according staff.
To monitor the connected charging infrastructure and identify outtakes due to technical errors, the caretakers interact with the system using the administration dashboard via a personal workstation in their office. In cases a computer is not accessible, they utilize the mobile application for checking the status of the charge points.
Beside the functionality as a supervisory system for the infrastructure, the caretakers need access to the reservations made on the \acrshortpl{cs} to resolve conflicts related to system outtakes or managing the corresponding users.

\subsubsection{Administration}
\label{ch:Requirements Engineering:sec:Stakeholders:ssec:Staff:sssec:Administration}

Representing the administrative counterpart to the maintenance personal, the group of secretaries working at the administration office to manage internal data regarding staff and students belonging to the \acrshort{hka}, internal workflows or the communication with other organizations. 
This includes paperwork in relation to certain approvals for new students, maintaining the stored records and processing incoming requests. Among other things, they are responsible for the organizing of events on and around the campus sites, plan upcoming visitations from external facilities and coordinate the utilization and maintenance tasks for the charging infrastructure.
As a supervisory unit for the \nameref{ch:Requirements Engineering:sec:Stakeholders:ssec:Staff:sssec:Maintenance Personal}, they are scheduling maintenance and instructing the caretakers in case of extraordinary servicing.  
Therefore, staff members attending these positions need access to the \acrshort{csms} similar to the scope of the maintenance personal. In contrast to the caretakers, they are not leaving their office for interaction with the \acrshortpl{cs}, therefore they do not depend on the functionalities the mobile application has to offer.

\section{Personas}
\label{ch:Requirements Engineering:sec:Personas}

Appropriate to the defined stakeholders in the previous section \ref{ch:Requirements Engineering:sec:Stakeholders}, each stakeholder requires a representation as persona in regards to a dedicated user account in front of the system.
This allows a mapping between the accounts and the existing system roles, which enables easier separation between the use cases extracted later on and the assignment to the different stakeholders. 
Hereinafter, personas prescribing individuals inside the scenario including their dedicated demands in respect of the challenges their particular stakeholder group has to face. 

\subsection{Lisa Knaus}
\label{ch:Requirements Engineering:sec:Personas:ssec:Lisa Knaus}

Lisa Knaus is a \nameref{ch:Requirements Engineering:sec:Stakeholders:ssec:Student} at the \acrshort{hka} and is studying computer science in third semester. In respect of her current living situation, located at a village with limited access to public transportation, she primarily relies on driving by car as main transportation mean.
Fostered through financial benefits promoted by the different car manufacturers for young drivers with interest in acquiring an \acrshort{ev}, she sold her old car with \acrshort{ice} and switched to a \acrfull{fev}. 
Under normal circumstances the health of the internal battery allows her to drive with a full charge from her hometown to the main campus site and back without a need to recharge. Occasionally she forgot to recharge at home and need a charging possibility for her vehicle during the lecture time.
Depending on the day and the according demand of the charging infrastructure, this could be a challenging task and she needs to arrive very early to get a free charging spot.

\subsection{Holger Starke}
\label{ch:Requirements Engineering:sec:Personas:ssec:Holger Starke}

Holger Starke is an employee working as member of the \nameref{ch:Requirements Engineering:sec:Stakeholders:ssec:Staff} at \acrshort{hka}. Within the in-house server administration team his daily tasks are covering the maintenance of the server landscape, applying new versions of internal applications and the creation of backups securing the records of the internal user databases. 
Thanks to the subsidies for electric vehicles from the state and his employer, he has bought a \acrshort{fev} and used it as his main means of transport during this time.
For compensation of the still missing charging possibilities at his apartment near the institution, he drive the short distance with his car every day to use the existing charging infrastructure at the \acrshort{hka} for recharge the batteries during working hours.
Therefore, he has to use the available \acrshortpl{cs} at the different campus locations without any guarantee to recharge. With the increasing amount of students and coworkers arriving with \acrshortpl{ev} over the time, the search for a free \acrshort{cs} becomes more and more demanding.
Especially scheduled service work later in the day located at other campuses emerge to a running the gauntlet for parking lots with \acrshortpl{cs} equipped. 

\subsection{Dieter Krause}
\label{ch:Requirements Engineering:sec:Personas:ssec:Dieter Krause}

Dieter Krause is engaged in the role of a caretaker as part of the \nameref{ch:Requirements Engineering:sec:Stakeholders:ssec:Staff:sssec:Maintenance Personal} at \acrshort{hka}. During his previous position he acted as carrier for the daily book orders for the students at the university library on the main campus site. 
For the reduction of \acrshort{co2} the \acrshort{hka} bought a \acrshort{fev} for bridging the short distances between the various libraries within the city.
This confronted him with the according up- and downside of driving an \acrshort{ev} in a very early stage.
Missing charging opportunities and the insufficient battery life were only two obstacles he has to bother with every day. This led him to the conclusion to switch positions and he took over the responsibility for the charging infrastructure maintenance and management on the campus sites to improve the situation for all involved parties.
Alongside the continuously increasing number of \acrshortpl{cs}, he have fostered the establishment of dedicated charging possibilities for the members of the \nameref{ch:Requirements Engineering:sec:Stakeholders:ssec:Staff} and introduced a exclusive \acrshort{cs} for the \acrshortpl{ev} of the \nameref{ch:Requirements Engineering:sec:Stakeholders:ssec:Staff:sssec:Maintenance Personal}. 
Even at days with high emergence of student vehicles on the public parking lots, he and his colleagues could recharge the batteries of their \acrshortpl{fev}. 

\subsection{Nadine Funke}
\label{ch:Requirements Engineering:sec:Personas:ssec:Nadine Funke}

Nadine Funke act as a secretary in the \nameref{ch:Requirements Engineering:sec:Stakeholders:ssec:Staff:sssec:Administration} office  at \acrshort{hka}. Formerly she worked at the public administration office in Karlsruhe in the area of public parking spaces, where she has made her first experience with the problems regarding the management of the charging infrastructure on public parking lots.
For this reason she does not drive a \acrshort{ev} herself and uses public transportation instead. In contrast to her previous position at the public administration office, the tasks regarding the management of the \acrshortpl{cs} are thanks to the introduced system much more convenient and do not require a lot of manual work.
By accessing the \acrshort{csms} dashboard, she could filter for the wanted \acrshort{cs} and have all the information in one place.
However, blocking \acrshortpl{cs} for upcoming visitors are, despite the new system, still a huge overhead for the secretaries at the \acrshort{hka}.
Up to now, she has had to select the specific \acrshort{cs} and the corresponding connector to block it, and she has to unblock it again when the guest has arrived.

\section{Role Mapping}
\label{ch:Requirements Engineering:sec:Role Mapping}

To distinguish between the single users and their privileges in compliance with the range of associated functionalities, the system introduces a set of roles. Beside the role of the basic user, two variations of administrators and a user for demo purposes are provided.
For a better understanding of these privileges and their purpose inside the system, the following table \ref{tab:system-role-collection} lists the relevant roles, their system shortages and a brief explanation.

\begingroup
\setlength{\tabcolsep}{10pt} % Default value: 6pt
\renewcommand{\arraystretch}{1.5} % Default value: 1
\begin{table}[!ht]
    \centering
    \caption{Role collection provided by the system}
    \begin{tabular}{c|c|c|m{6cm}}
        Role & System & Shortage & Description \\
        \hline
        Basic & \verb|BASIC| & \verb|B| & Standard user without administrative privileges \\
        Admin & \verb|ADMIN| & \verb|A| & User with administrative privileges inside one organization \\
        Site Admin & \verb|BASIC| & \verb|B| & User with administrative privileges for assigned sites \\
        Site Owner & \verb|BASIC| & \verb|B| & User with administrative privileges for owned sites \\
        Super Admin & \verb|SUPER_ADMIN| & \verb|S| & User with extended administrative privileges for the management of organizations and tenants \\
        Demo & \verb|DEMO| & \verb|D| & Demo user with predefined login data, primarily used for demonstration purposes \\
    \end{tabular}
    \label{tab:system-role-collection}
\end{table}
\endgroup

Based on the available user privileges introduced in \ref{tab:system-role-collection} and provided by the system out of the box, the next step is the mapping of the created personas, established in section \ref{ch:Requirements Engineering:sec:Personas}, to their corresponding role inside the system.
The criteria for the assignment is primarily based on the tasks the single users have to fulfill and in respect of their position inside the scenario.

\begingroup
\setlength{\tabcolsep}{10pt} % Default value: 6pt
\renewcommand{\arraystretch}{1.5} % Default value: 1
\begin{table}[!ht]
    \centering
    \caption{Role mapping of the system roles to the different groups identified in the scenario described in \ref{ch:Requirements Engineering:sec:Scenario}}
    \begin{tabular}{c|c}
        Group & Role \\
        \hline
        \nameref{ch:Requirements Engineering:sec:Stakeholders:ssec:Student} & Basic \\
        \nameref{ch:Requirements Engineering:sec:Stakeholders:ssec:Staff} & Basic \\
        \nameref{ch:Requirements Engineering:sec:Stakeholders:ssec:Staff:sssec:Maintenance Personal} & Admin \\
        \nameref{ch:Requirements Engineering:sec:Stakeholders:ssec:Staff:sssec:Administration} & Admin
    \end{tabular}
    \label{tab:stakeholder-role-mapping}
\end{table}
\endgroup

\section{Goals}
\label{ch:Requirements Engineering:sec:Goals}

Based on the scenario and the corresponding stakeholders, the relevant goals for the resulting implementation are prescribed. As part of this work, a goal is handled as a requirement, the final product has to fulfill. 
Taking into account the different wishes regarding the situations in which the respective stakeholders interact with the system, the following objectives could be defined.

\noindent \textbf{Goal 1 - Management Capabilities} As a major concern of this work the designed system, should serve as a central point for management purposes of the registered \acrshortpl{cs}. This should provide a basic set of functionalities covering specific administration tasks tailored for a certain group of users. 
Beside a role mapping according to a subset of functions, the system should detect unintended usage and prohibit breach of privileges.
Furthermore, the involved users will be provided with all relevant information by the system and know their current position within the different phases of the processes at each time. 

\noindent \textbf{Goal 2 - Self-Healing and Autonomous Processes} In case of invalid user input or violation of predefined constraints inside the processes, the system should provide mechanisms for self-healing and autonomous error handling. 
In addition to increased consistency of data within the underlying database, the ease of use of the system should be a positive side effect.

\noindent \textbf{Goal 3 - Support of relevant standards} Considering the implementation of relevant communication protocols and the interoperability between different systems of \acrshortpl{cs}, \acrshortpl{ev} and \acrshortpl{csms}, the system should support well-established standards for communication and data exchange in the context of \acrshort{emobility}. 
Alongside the standards like \acrshort{ocpp} or \acrshort{ocpi}, described in the subsection \ref{ch:Fundamentals:sec:Electric Mobility:ssec:Relevant Standards}, other alternatives exist in the further evolving industry with different targets in mind. 
Considering the already implemented standard inside the underlying product, the \nameref{ch:Fundamentals:sec:Electric Mobility:ssec:Relevant Standards:sssec:OCPP} should be the protocol for the communication between \acrshort{cs} and \acrshort{csms}.

\noindent \textbf{Goal 4 - Modular Design} In addition to the previously mentioned goals that focus on the implemented functionality, the architectural design of the software itself should also be considered. 
As part of a system that follows both a monolithic and a microservice-based approach, multiple services interact simultaneously alongside each other. The implementation and software architecture of this work aims to avoid compromising existing functionality through poor design choices or neglected separation of concerns. 
Preferably, the new functionality should be encapsulated as part of a module, which separates the system based on its responsibilities according to the definition in \cite{clements_documenting_2011}. This should ensure easier extensibility of existing features and allow convenient maintainability.

\section{Use Cases}
\label{ch:Requirements Engineering:sec:Use Cases}

The use cases illustrated in the use case diagram in \ref{fig:use-cases} were elaborated based on the goal definitions in the previous section \ref{ch:Requirements Engineering:sec:Goals} and the stakeholder requirements from section \ref{ch:Requirements Engineering:sec:Stakeholders}.
Typically, software development processes such as \acrshort{rup} \cite{kruchten_rational_1999} differ between the groups of stakeholders who have an interest in the resulting product and the actors who are actually interacting with the real system. 
As part of this scenario, the stakeholders actually represent the actors used for defining the use case in regard of interacting with the system.
The main use case, as highlighted by \textbf{goal 1}, involves managing and administering reservations within the reservation system. This encompasses creating, modifying, viewing and deleting existing reservations made either by the users themselves or on their behalf by the administrator.
Corresponding to \textbf{goal 2}, the \textit{Scheduler} entity facilitates the required self-healing procedures. This necessitates the system to communicate with itself and employ background processes to manage forthcoming, current, and expiring reservations.
Apart from the primary focus on non-functional requirements related to \textbf{goals 3} and \textbf{4}, an additional functional requirement regarding the selective activation and deactivation of the reservation module could be identified.
The privileges for maintenance and configuration of the particular organizations, represented as tenants inside the system, is part of the \verb|SUPER_ADMIN| role.
Usually, an external service provider outside the customer organizations is responsible for determining the scope of operation for the service offerings provided to its customers. This role is referred to as \verb|SUPER_ADMIN|. As there is no direct interaction with the system regarding the reservation process, there is no dedicated actor assigned to this role within the above scenario.
For comprehensively addressing all the aspects mentioned in the set goals, it is necessary to include this role with its corresponding functionality when defining the required features.

\begin{figure}[!ht]
    \centering
    \includegraphics[scale=0.4]{resources/images/main/2_requirements_engineering/UseCases.png}
    \caption{Use Cases regarding the stakeholders of the reservation system inside 'Open e-Mobility' \cite{noauthor_open_2023}}
    \label{fig:use-cases}
\end{figure}

%% LaTeX2e class for student theses
%% sections/main/3_approach.tex
%%
%% Karlsruhe University of Applied Sciences
%% Faculty of  Computer Science and Business Information Systems
%%
%% --------------------------------------------------------
%% | Derived from sdqthesis by Erik Burger burger@kit.edu |
%% --------------------------------------------------------


\chapter{Approach}
\label{ch:Approach}

This chapter describes the order and duration for processing the identified tasks, in accordance with the design considerations outlined in the previous chapter \ref{ch:Requirements Engineering}. 
For this purpose, this work is divided into two parts, which will later serve to classify the aforementioned challenges.
The first part is theoretical, including the paper's preparation and research to understand the problem. 
The second part afterwards is practical, encompassing the tasks needed to implement the system. \\
\noindent In order to aid planning, timestamps are estimated for each task or logical unit of tasks to provide reference points during the processing of corresponding dedicated work packages. 
A logical unit, as used here, refers to a coherent cluster of individual tasks which share content or functionality and can be worked on simultaneously. Working on one task without the others would be less significant.
The standard duration for processing a task or unit is set at two weeks (10 days) typically, to ensure sufficient time for completion. The duration may differ in certain cases and is determined based on the relevance and volume of related challenges. 
After describing the two conceptual parts alongside their respective tasks and processing times, this chapter concludes with a comparison of the planned procedure with the actual one.

\section{Theoretical Part}
\label{ch:Approach:sec:Theoretical Part}

With regard to the delimitation of the problem space and the synthesis of the central problem, the main aspect of the theoretical part of this work consists of a comprehensive literature review.
This involves analyzing existing work on reservation processes used for managing \acrshortpl{cs} and allocating charging infrastructure to a specific \acrshort{evu} within a limited time frame, using various methods.
With regards to the default processing time mentioned earlier, the duration for compilation as well as analysing articles and field projects related to this area is set to one month. 
The expected result should be used as the groundwork for the subsequent practical part, which is summarised in chapter \ref{ch:Literature Review}.
As well as examining the current state of reservation systems, this review process defines a boundary for the necessary fundamentals of contextualisation, resulting in the chapter \ref{ch:Fundamentals}.
Therefore, a systematic requirements engineering, taking into account the facts gathered during the research phase and the collection of baselines, using a customised scenario to select relevant use cases, results in the chapter \ref{ch:Requirements Engineering}.
Chapter \ref{ch:Design} presents the design phase, which follows the summarized requirements in chapter \ref{ch:Requirements Engineering} and the results of the literature review in chapter \ref{ch:Literature Review}. During this phase, mockups of the relevant interfaces as well as the underlying processes will be created.
With the help of these designed concepts, the actual implementation of the reservation module resulting in chapter \ref{ch:Implementation} takes place.
Subsequently, the theoretical section concludes by validating the achieved functionality and presenting the final results, culminating in \ref{ch:Analysis and Validation}. Potential avenues for further research are also highlighted.

\section{Practical Part}
\label{ch:Approach:sec:Practical Part}

The practical part utilizes the results of the theoretical phase, as described in section \ref{ch:Approach:sec:Theoretical Part}. Starting with the analysis of the current state of the applications used for the development of the prototypical reservation approach of this work, the final step is the implementation of the solution, taking into account the elaborated constraints and requirements.
Seven iterations covering the implementation steps for the use cases described in Section \ref{ch:Requirements Engineering:sec:Use Cases} were conducted, commencing from the end of the theoretical phase that includes four weeks of literature research and information gathering, until the end of the processing stage.
This allows for a six-week buffer to address any unforeseen features, correct any issues with the implemented logic, and complete the documentation tasks related to the theoretical part.
Following this, a list summarising the scheduled iterations and associated timeframes will be provided.

\begin{multicols}{2}
\begin{description}
    \item[Iteration 1] 18.04.2023 - 01.05.2023
    \item[Iteration 2] 02.05.2023 - 15.05.2023
    \item[Iteration 3] 16.05.2023 - 29.05.2023
    \item[Iteration 4] 30.05.2023 - 12.06.2023
\end{description}
\begin{description}
    \item[Iteration 5] 13.06.2023 - 26.06.2023
    \item[Iteration 6] 27.06.2023 - 10.07.2023
    \item[Iteration 7] 11.07.2023 - 24.07.2023
\end{description}
\end{multicols}

\noindent According to this iterative model, a functional increment of the resulting software should be provided after each iteration.

\section{Actual Course Of Action}
\label{ch:Approach:sec:Actual Course Of Action}

As there is a limited amount of available research and existing projects on reservation features for \acrshortpl{cs}, the analysis and setup of the 'Open e-Mobility' \cite{noauthor_open_2023} solution and its particular components commenced a day earlier than planned.
Consequently, some adjustments have been made to the above timescales for the general processing of the relevant parts.
Beginning with the identification of missing functionality required for the subsequent implementation of the customised processes.
The first step was to investigate the 'e-mobility charging stations simulator' \cite{noauthor_julianhbuechere-mobility-charging-stations-simulator_nodate} for local simulation of a customizable charging infrastructure.
Because of the identified feature gap regarding missing reservation capabilities conforming with \acrshort{ocpp} 1.6 \cite{noauthor_ocpp_nodate-1}, the implementation of this crucial feature for testing and evaluation purposes was included as preliminary work for the 'Open e-Mobility' team.
After the team members had successfully reviewed the implemented logic and minor reworks of the code base, the changes were merged into the standard product.
Now, the extended \acrshort{cs} simulator was then used as a reservable charging infrastructure for the subsequent iterations.
Particularly, to validate the implementation of reservations compliant with version 1.6 of the \acrshort{ocpp} protocol, considering both pre-existing functionalities in the backend application and the adaptations according to this work.
This included not only the backend development, but also the implementations within the frontend application during iterations three and four.
Due to the partially existing features mentioned above, the third iteration takes less time than expected, resulting in an earlier commencement of the relevant functions in the web frontend.
In light of the time savings achieved in the preceding iterations, the fifth iteration for implementing the custom reservation function began prematurely as well. 
To account for the unexpected scope of required operations alongside an end-to-end development approach, the author decided to merge the fifth and sixth iterations, resulting in an extended fifth iteration and a total of six iterations instead of the originally planned seven iterations.
Because of the previous schedule adjustments, the final iteration also started earlier and completed later than intended. 
Primarily referring to the incidence of bugs and further requirements to manage reservations internally.
The resulting schedule, including the iterations, their specific duration, and the corresponding tasks, can be found in table \ref{tab:development-iterations} below.

\begingroup
\setlength{\tabcolsep}{10pt} % Default value: 6pt
\renewcommand{\arraystretch}{1.5} % Default value: 1
\begin{table}[!ht]
    \centering
    \caption{Schedule for implementing the reservation feature designed in this study}
    \begin{tabular}{c|c|c|m{7cm}}
        Iteration & From & To & Objective \\
        \hline
        1 & 17.04.2023 & 28.04.2023 & Feature gap detection within the existing application \\
        2 & 01.05.2023 & 12.05.2023 & Adjustments to the 'e-mobility charging stations simulator' \cite{noauthor_julianhbuechere-mobility-charging-stations-simulator_nodate} to support reservations complying to \acrshort{ocpp} version 1.6 \\
        3 & 15.05.2023 & 19.05.2023 & Implementation of \acrshort{ocpp} 1.6 reservation functionality in the backend application \\
        4 & 22.05.2023 & 02.06.2023 & Implementation of \acrshort{ocpp} 1.6 reservation functionality in the web frontend application \\
        5 & 05.06.2023 & 30.06.2023 & Implementation of customized reservation process in the frontend and backend applications \\
        6 & 03.07.2023 & 28.07.2023 & Implementation of \acrshort{ocpp} 1.6 reservation functionality and the customized reservation process inside the mobile application \\
    \end{tabular}
    \label{tab:development-iterations}
\end{table}
\endgroup

As well as implementing the reservation system features, this work covered other tasks related to 'Green4EVer' \cite{noauthor_hka_nodate} project, although they were only indirectly connected to this work.
The decision to include these tasks is primarily based on the simplification of future deployment scenarios, such as demonstration or test cases, which require an automated and reliable setup.
Therefore, this section mentions these tasks in addition to a brief explanation of the challenge. A detailed description is omitted from both the subsequent 'Design' chapter \ref{ch:Design} and the 'Implementation' chapter \ref{ch:Implementation} according no direct relation to the primary problem that this work covers.

\begin{description}
    \item[Task 1 - MongoDB Manifests] As the public repository offers no other deployment options, the only available one was containerization and operation using the Docker Compose orchestration tool \cite{noauthor_overview_2023}. 
    To offer a more scalable and robust solution, the industry standard for deploying containerized applications therefore recommends employing \acrshort{k8s} \cite{noauthor_produktionsreife_nodate} as sophisticated orchestrator. This led to this first task. 
    Nevertheless, the manifest files essential for creating a 'StatefulSets' \cite{noauthor_statefulsets_nodate} deployment, a standard way for deploying database applications inside \acrshort{k8s}, were still not yet included in the current project. 
    Hence, the MongoDB \cite{noauthor_mongodb_nodate} database's custom setup, including test data and users, already existing as part of the local setup, was employed as a blueprint for creating the necessary manifests required for the \acrshort{k8s} deployment.
    \item[Task 2 - MongoDB Helm Chart] To further reduce the effort required to deploy the database application to \acrshort{k8s} instances, the Helm \cite{noauthor_helm_nodate} toolbox and its charts for encapsulating \acrshort{k8s} manifests are integrated.  
    From the manifest files created in \textbf{task 1}, the corresponding Helm charts for a deployment are created, providing parameterization of a namespace and all required \acrshort{k8s} resources, such as 'PersistentVolumes' \cite{noauthor_persistent_nodate}, 'ConfigMaps' \cite{noauthor_configmaps_nodate} or 'Namespaces' \cite{noauthor_namespaces_nodate}. 
    \item[Task 3 - \acrshort{k8s} Deployment] For the integration of the remaining applications of the 'Open e-Mobility' \cite{noauthor_open_2023} solution, additional Helm charts could be organized for the backend and the frontend application. 
    Regarding the limited resources provided by the used orchestration service, the new charts required further modifications to integrate with the MongoDB chart and the deployment restrictions of the project. This involves a shift from the predefined microservice-based deployment strategy to a monolithic approach resulting in reduced resource allocation.
    \item[Task 4 - Ingress Deployment \& Public Availability] In order to access the applications within the \acrshort{k8s} cluster, it is necessary to use an 'Ingress' \cite{noauthor_ingress_nodate} acting as a reverse proxy. 
    Therefore, this ingress must be located within the 'Namespace' the application lives in.
    Alongside the charts for the backend and frontend mentioned in \textbf{task 3}, the collection includes a configuration for deploying an NGINX ingress controller \cite{noauthor_nginx_nodate} using the \acrshort{k8s} ingress class. 
    In combination with a public \acrshort{dn} and the corresponding \acrshort{ip} address, the final stage of the \acrshort{k8s} deployment consisted of configuring the ingress deployment and registering it at a \acrshort{dn} service.
\end{description}

As a result, the \acrshort{k8s} setup shown in \ref{fig:k8s-setup}, considers a deployment scenario that includes a custom namespace, referred to in this project as \verb|ev|, containing the particular applications, an NGINX ingress controller with a publicly accessible \acrshort{dn}, and the MongoDB database with built-in reinitialization of custom test data and users.
In terms of application distribution, only one node is used with multiple pods within the \acrshort{k8s} cluster.

\begin{figure}[!ht]
    \centering
    \includegraphics[scale=0.4]{resources/images/main/3_approach/KubernetesDeployment.png}
    \caption{\acrshort{k8s} deployment architecture for the 'Open e-Mobility' \cite{noauthor_open_2023} solution}
    \label{fig:k8s-setup}
\end{figure}

%% LaTeX2e class for student theses
%% sections/main/4_literature_review.tex
%%
%% Karlsruhe University of Applied Sciences
%% Faculty of  Computer Science and Business Information Systems
%%
%% --------------------------------------------------------
%% | Derived from sdqthesis by Erik Burger burger@kit.edu |
%% --------------------------------------------------------


\chapter{Literature Review}
\label{ch:Literature Review}

As the primary part of the theoretical phase described in section \ref{ch:Approach:sec:Theoretical Part}, this chapter provides an overview of research related to the problem statement of this particular work.
Based on the existing studies, the current state of reservation systems for charging infrastructure management is identified and outlined in the last section \ref{ch:Literature Review:sec:Current State} of this chapter.
Given the current state of implemented systems of this kind and the results of the available research, the design part in chapter \ref{ch:Design} will use these results for the design of the system and corresponding processes implemented in this thesis.

\section{Related Work}
\label{ch:Literature Review:sec:Related Work}

In the context of supporting the wider acceptance of \acrshortpl{ev} in the society, its decarbonization and the reduction of air pollution caused by vehicles using \acrshortpl{ice} \cite{basmadjian_reference_2020}, the integration of \acrshortpl{ev} in the daily lives of citizens is still in ongoing and far from being complete.
Obstacles like insufficient charging infrastructure in comparison to the availability of gas stations, long charging times and high investments for the public and private sectors \cite{basmadjian_reference_2020,orcioni_ev_2020} are only a few examples mitigating the end users interest in switching to \acrshortpl{fev}.
For this reason, the surrounding \acrshort{emobility} ecosystem including the software for the interaction and management of the user with the charging infrastructure and the convenience of the underlying processes used for the implementation is an essential part.
Given the system landscape and protocols employed in these scenarios, the ability to reserve a \acrshort{cs} beforehand is usually lacking and has not yet been integrated into the current implementations of \acrshortpl{csms}.
As part of the comprehensive literature review in this thesis, the following selected studies describe methods for applying reservation approaches and their respective processes to current implementations. This is intended to provide an extension to the existing system landscapes and protocols in this field, with the aim of achieving certain objectives. \\
%% --------------------------------------------------------------------------------------------------------------------------------
%% An Interoperable Reservation System for Public Electric Vehicle Charging Stations: A Case Study in Germany
%% --------------------------------------------------------------------------------------------------------------------------------
\noindent In \cite{basmadjian_interoperable_2019}, the authors argued that the reservation of charging infrastructure, especially for \acrshortpl{cs}, represent a pivotal role in a seamlessly integration of the \acrshort{cs} into the transportation and mobility sector.
Apart from the basic requirements for the design and implementation of an interoperable reservation system, they provide a breakdown of the different reservation approaches into four types of reservations. 
For designing their system, Basmadjian et al. used the \acrshort{emsa} \cite{kirpes_e-mobility_2019} framework as a blueprint for the system model and its engineering and implemented on of their reservation types as \acrshort{poc} in a showcase located in Bavaria, Germany.
Proclaiming being the first contribution regarding the proposal of an interoperable reservation system for \acrshort{ev} charging, this study introduced the following four different kinds of reservations. Primarily based on the \textbf{start time} and the \textbf{reliability} a reservation takes place, a differentiation between \textit{Uncertain Ad-Hoc}, \textit{Guaranteed Ad-Hoc}, \textit{Uncertain Planned} and \textit{Guaranteed Planned} is provided.
The \textit{Planned} reservation, in contrast to \textit{Ad-Hoc} reservations, allows the \acrshort{evu} to pre-block the respective connector, requiring start and end timestamps for the reservation, instead of the immediate blocking of a \acrshort{cs} during a pre-configured time span. 
Moreover, when taking into account the reliability of a reservation and the corresponding charging stations, \textit{uncertain} reservations are not able to guarantee a free parking space at the specified connector, whereas \textit{guaranteed} ones are able to do so.
Using the \textit{planned} reservation method makes it possible to make multiple reservations for one \acrshort{cs}, unlike the \textit{ad-hoc} setup where only one reservation is allowed.
From the requirements that the authors gathered during the design process of their study, such as improving the short-term planning capabilities of \acrshortpl{emsp} and \acrshortpl{cso} or increasing the fare convenience for the \acrshortpl{evu}, their work resulted in the implementation of an uncertain ad-hoc reservation system, including a mobile application for the \acrshort{evu}, and the required back-end services for communication and data exchange with the \acrshortpl{cs}.
Concerning data exchange and modelling, they relied on the \acrshort{ocpp} protocol and the \acrshort{iso} standard 15118 to facilitate standard communication between the system and the infrastructure.
In addition to the proposed implementation, Basmadjian et al. reviewed available research on reservation systems for \acrshortpl{cs} as part of the literature review conducted in this study.
In the course of their research, the authors observed that the majority of studies solely focus on improving the satisfaction of \acrshortpl{evu} by reducing their wait time and associated loading costs, while simultaneously maximising the utilisation of \acrshortpl{cs}.
Therefore, these papers applied algorithmic approaches for efficient scheduling of reservations like in \cite{kim_efficient_2010,xiang_reservation-based_2011,qin_charging_2011} and do not provide sufficient and interoperable solution for reserving charging infrastructure emphasizing the need for standard communication protocols like \acrshort{ocpp} or common data models. 
Consequently, the authors concluded that unlike other sectors, the implementation and analysis of reservations for \acrshortpl{cs} and \acrshortpl{ev} is still in a evolving stage. \\
%% --------------------------------------------------------------------------------------------------------------------------------
%% An OCPP-Based Approach for Electric Vehicle Charging Management
%% --------------------------------------------------------------------------------------------------------------------------------
\noindent Apart from reservation systems, which aim to manage the charging infrastructure through the targeted input of an \acrshort{evu}, other approaches, such as in \cite{hsaini_ocpp-based_2022}, have tackled the problem of \acrshort{cs} management in a more intelligent and automated way.
Hsaini et al. also recognized the issue that the current versions of \acrshort{ocpp} enable reservations only at the time of booking through the 'ReserveNow' operation \cite{noauthor_ocpp_nodate-1,noauthor_ocpp_nodate}.
Therefore, the authors want to improve the predefined reservation function to allow the \acrshortpl{evu} to reserve a \acrshort{cs} in advance, which resulted in a mobile application for the \acrshortpl{evu} for making the reservation, an algorithm for optimizing the charging schedule, a web application for monitoring purposes and the according backend, executing the operations based on \acrshort{ocpp}.
For creating the according reservations, the \acrshort{evu} has to specify the arrival and departure time, the desired amount of energy, battery capacity, and optionally the initial state of charge with the desired state of charge by the mobile application.
Once the optimization algorithm has been run in the backend application considering the provided user preferences, a list of available \acrshortpl{cs}, along with the available energy and corresponding electricity costs, is presented to the user. For a successful reservation the selection of the \acrshort{cs} and the provided time must be confirmed. \\
%% --------------------------------------------------------------------------------------------------------------------------------
%% EV Smart Charging with Advance Reservation Extension to the OCPP Standard
%% --------------------------------------------------------------------------------------------------------------------------------
\noindent A comparable approach is presented in \cite{orcioni_ev_2020}. By utilizing a mobile application, users can request a booking by defining their preferences and specifying their flexibility options regarding in terms of arrival time, battery's \acrshort{soc} and desired final charge.
The system presents a listing of bookable slots that are generated based on the findings of an optimization algorithm. This enables the user to finalize the booking by adjusting the provided parameters through negotiation with the system itself. \\
%% --------------------------------------------------------------------------------------------------------------------------------
%% Charging reservation service for electric vehicles using automatic notification
%% --------------------------------------------------------------------------------------------------------------------------------
\noindent Unlike the reservation systems mentioned earlier, \cite{zarkeshev_charging_2018} suggests a solution that eliminates the \acrshort{evu}'s interaction until a certain point in the reservation process.
Therefore, Zarkeshev and Csiszar developed a concept based on an automated reservation process that only requires notification of the user, using real-time information about the vehicle's status and location.
Primarily designed for the usage in smart cities and scenarios alongside the highway, each \acrshort{cs} takes the role of a server with a pre-defined radius of operation. By driving through such a zone, the \acrshort{cs} automatically identifies the \acrshort{ev} and its battery level, which results in a automatic notification with a proposal to make a recharge in cases of low charging level. 
Assuming the driver accepts the notification, the system will automatically reserve a charging slot at the \acrshort{cs}. The system will assign a charging time based on the availability of the \acrshort{cs} and the duration of the charge. These factors will be considered in relation to the status of the battery.
Moreover, the server acts as a monitor, keeping track of all bookings and notifying the relevant drivers in the case of delayed arrivals, which could lead to cancellations or postponements of bookings, resulting in adjustments to the overall schedule. \\
%% --------------------------------------------------------------------------------------------------------------------------------
%% Electric vehicles charging reservation based on OCPP
%% --------------------------------------------------------------------------------------------------------------------------------
\noindent When considering the communication and information exchange exclusively between the \acrshortpl{ev} and the \acrshortpl{cs}, implementing charging management into a smart grid system alongside scenarios like \acrshort{v2g} could be a next possible step. 
By using reservations, the solution presented in \cite{orcioni_electric_2018} proposes opportunities to optimize resources on both the power grid side and the \acrshort{evu} side.
In addition to previous solutions, Orcioni et al. developed their approach by extending the \acrshort{ocpp} protocol with prescribed reservation functionality.
This resulted in a mobile application for end-users, in which the related \acrshort{ocpp} operations are implemented in the backend service, allowing users to reserve a \acrshort{cs} in advance by negotiating charging parameters such as arrival time, duration, location, price, percentage of final charge and the required power.
Moreover, a data model is provided that considers these parameters. \\
%% --------------------------------------------------------------------------------------------------------------------------------
%% SGAM-Based Analysis for the Capacity Optimization of Smart Grids Utilizing e-Mobility: The Use Case of Booking a Charge Session
%% --------------------------------------------------------------------------------------------------------------------------------
\noindent Based on smart grid integration approaches like \acrshort{v2g}, other studies like \cite{garcia_sgam-based_2023} utilize the reservation approach for the design of a flexible user-centric architecture for capacity optimization of the underlying power grid, considering the energy requirement of the grid and its capacity restrictions.
Therefore, Garcia et al. applied the \acrshort{sgam} \cite{noauthor_sgam_nodate} methodologies to provide a potential implementation of \acrshort{emobility} as a distributed storage asset, also known as \acrshort{der}.
Unlike the objectives of the previous studies, the modelling for booking a charge session through a mobile application, targets the balance of \acrshort{res}, along with the discharge of \acrshortpl{ev} compensating power peaks during times of high-demand. 
As well as the other implementations, this system also based on the \acrshort{ocpp} protocol, and establishes a reservation process that locks the \acrshort{cs} after selection through an appropriate interface by the backend system. Unlocking takes place upon arrival and successful authorization via a token or in case of non-arrival of the driver. \\
%% --------------------------------------------------------------------------------------------------------------------------------
%% An Efficient Scheduling Scheme on Charging Stations for Smart Transportation
%% --------------------------------------------------------------------------------------------------------------------------------
\noindent Other algorithmic approaches using reservation processes to reduce the cost of charging or to increase consumer satisfaction of \acrshortpl{ev}, for example, also rely on reservation-based scheduling schemes.
In the paper \cite{kim_efficient_2010}, for example, the authors propose a method for \acrshort{cs} to decide the service order of multiple requests, prohibit the introduction of additional systems altogether, and try to extend the functionality of \acrshort{cs} itself. 
To address this, Kim et al. developed a linear rank function based on the estimated arrival time, waiting time and amount of power required to calculate the order of charging sessions.
As a result, the person requesting the reservation may choose to take advantage of the opportunity to recharge, or may have to move to another station altogether. \\
%% --------------------------------------------------------------------------------------------------------------------------------
%% Electric Vehicle Smart Charging Reservation Algorithm
%% --------------------------------------------------------------------------------------------------------------------------------
\noindent Furthermore, in \cite{flocea_electric_2022} Flocea et al. addressed the issue of uncertain availability of \acrshort{cs} along the route of \acrshortpl{evu}. 
This proposal aims to overcome the limitations of the 'ReserveNow' feature defined in the \acrshort{ocpp} protocol to enable drivers to plan longer trips by creating charging reservations for upcoming days.
Based on the \acrshortpl{cs} reservation and transaction history, this solution is backed by an algorithm that generates the corresponding reservations in the form of intervals to guarantee the availability of the reserved connector when it arrives at the station and to avoid possible overlaps of reservations.
To create a reservation, the user needs to choose a valid timeslot, including the start and end time, while preventing other \acrshortpl{evu} from charging their vehicles at the reserved \acrshort{cs}.
In order to manage reservations internally, a reservation is assigned to a particular status type, which describes its position within the reservation life cycle within the system. 
Apart from \textit{New}, \textit{In progress}, \textit{Completed}, the status \textit{Cancelled} is introduced, which allows the system to treat the reservation according to certain circumstances.
Moreover, the reservation algorithm could adjust the reservation according to the previous conditions and assume that the reserved time counts as charging time. When the reserved time comes to an end, charging will stop automatically. \\
%% --------------------------------------------------------------------------------------------------------------------------------
%% A Reference Architecture for Interoperable Reservation Systems in Electric Vehicle Charging
%% --------------------------------------------------------------------------------------------------------------------------------
\noindent In terms of a more elevated perspective on reservation systems and potential architectures considering ways to select required stakeholders and requirements, \cite{basmadjian_reference_2020} presents a reference architecture to conceive interoperable reservation systems.
The benefits of such architectures for future reservation systems in this context and the key stakeholders needed during the requirements engineering phase are listed by Basmadjian et al. in this paper.
Besides the identification of the necessary requirements of the aforementioned stakeholders, the designed reference architecture and a \acrshort{poc} developed using this design, the authors introduced a set of design parameters for reservation systems.
In order to achieve demand-side management and capacity planning of existing charging infrastructure through reservation systems, the following design parameters proposed by the authors must be considered: \textit{Enforceability}, \textit{Planning}, \textit{Fee}, \textit{Data Availability}, \textit{Roaming} and \textit{Scheduling}.
When it comes to parameter coverage, each system can only fulfill a certain subset. The selection of these subsets should be based on the objectives the reservation system aims to achieve, during the design stage.
As a result, different types of systems may be created, each tailored to a specific purpose or scenario. \\
%% --------------------------------------------------------------------------------------------------------------------------------
%% Mobile Charging as a Service: A Reservation-Based Approach
%% --------------------------------------------------------------------------------------------------------------------------------
\noindent In contrast to the fixed charging stations introduced in subsection \ref{ch:Fundamentals:sec:Electric Mobility:ssec:Charging Infrastructure} by the classification of available charging infrastructure in figure \ref{fig:charging-station-classification}, studies such as \cite{zhang_mobile_2020} consider reservation processes for the utilization of mobile charging stations using a reservation systems.
The authors offer a design approach for an intelligent mobile charging control mechanism for electric vehicles, where mobile charging is promoted as an alternative recharging solution using mobile plug-in chargers to facilitate on-site charging service scenarios.
Relying on a reservation-based scheduling scheme that approximates optimal solutions for mobile chargers circulating between parked vehicles with charging appointments, the aim is to use mobile vans with plug-in chargers as \acrshortpl{cs} to provide a versatile charging service.
With the introduction of reservations, accurate estimates of future charging demand can be made and the strain on charging infrastructure and lack of \acrshortpl{cs} in certain areas could be alleviated.

\section{Current State}
\label{ch:Literature Review:sec:Current State}

Considering the various processes and methods that propose approaches to managing \acrshortpl{cs} through reservations described in section \ref{ch:Literature Review:sec:Related Work}, the current state of reservation systems for charging infrastructure management is illustrated below. \\
% Similarities
\noindent Firstly, this section describes the similarities in the technologies used and the overlaps identified in process design and implementation.
Most of the investigated systems include, in addition to the mobile applications for the \acrshortpl{evu} usage, one or more backend services that handle the information exchange between the mobile frontend and the corresponding charging infrastructure.
For reservation creation and communication between the \acrshort{csms} and the \acrshort{cs}, the all systems rely on the \acrshort{ocpp} protocol as an open standard protocol for charging infrastructure communication.
In order to mitigate unforeseen circumstances affecting the reservation made, the systems found included background processes ranging from cancellation of the reservation after a certain period of time, to automatic adjustment of the underlying charging timetable to reschedule the upcoming reservations.
Moreover, certain methodologies suggest utilising physical obstacles, such as sensors or system-controlled blockers, to ensure the successful completion of the booking process. \\
% Differences
\noindent The most significant difference between the proposed solutions is the way a reservation is modelled as an entity within the system and the degree of interaction with the \acrshort{evu}.
From fully automated scheduling algorithms that create reservations based on calculated intervals using vehicle and location data, requiring only confirmation from the user to minimise the need for user intervention, to methods that require various user inputs such as actual battery status, estimated time of arrival or desired battery charge level at the end of the reservation.
Some proposals have combined both approaches and established negotiation processes that include game theoretical aspects to negotiate the reservation slot directly with the backend system or the \acrshort{cs}.
Therefore, various properties are needed to model the reservation system internally. The properties that most reservations naturally support are the corresponding \acrshort{cs} and connector, as defined in \acrshort{ocpp}, as well as the start and end time, which extends the aforementioned protocol and simplifies reservation scheduling for the system. \\
% Design Criteria
\noindent In spite of the fact mentioned in \cite{basmadjian_interoperable_2019}, that the lack of extensive research and implementations unlike in other sectors, reservation systems for \acrshortpl{cs} and \acrshortpl{ev} still lack general and generic design requirements.
Certain design criteria are identical across the existing implementations. This heavily relies on the required functionality and the objectives they have to fulfil.
These may include the availability of payment options and the integration of self-managed rescheduling algorithms to reduce overlapping reservation intervals, and could be located within each system in a similar manner as stated or declared. \\
% Results Considered
\noindent Based on the results obtained, such as the reduced travel time depending on the demand on the charging infrastructure and the type of reservation chosen, reservation systems do not directly increase the utilisation of the charging infrastructure.
Moreover, studies on alternative charging approaches like mobile \acrshortpl{cs}, that provide more flexible processes, have only just scratched the surface in the existing literature. Nor a focus on the interoperability of the resulting implementations could be identified. \\
% Missing Features
\noindent Concerning the architectural design of the systems, most of the reservation functionalities extend or integrate with the \acrshort{ocpp} standard and place their custom enhancements on top of it. Only in \cite{flocea_electric_2022} a reservation module is provided  as a separate functionality, co-existing with the \acrshort{ocpp} reservation functionality within the \acrshort{csms}.
In terms of features such as the implementation of recurring reservations for more than one charging session within a specified date range, or the use of role concepts to control infrastructure management to differentiate functionality according to specific roles within the system, there is no elaboration.
Moreover, there is no control flow available for a more detailed and precise regulation or differentiation of specific phases that a reservation could undergo. 

%% LaTeX2e class for student theses
%% sections/main/5_design.tex
%%
%% Karlsruhe University of Applied Sciences
%% Faculty of Computer Science and Business Information Systems
%%
%% --------------------------------------------------------
%% | Derived from sdqthesis by Erik Burger burger@kit.edu |
%% --------------------------------------------------------

\chapter{Design}
\label{ch:Design}

This chapter presents a design that takes into account the processes cross-checked with existing approaches for reservation systems in the context of \acrshort{emobility}, as discussed in chapter \ref{ch:Literature Review}. Therefore, it relies on the previously stated definition of the requirements and necessary use cases, mentioned in chapter \ref{ch:Requirements Engineering}.
Besides conceptualizing required entities such as the reservation itself, the functionality for sufficient management capabilities of the underlying charging infrastructure is considered.
Furthermore, the mockups created during the theoretical part of this work are associated with the corresponding capabilities that the system has to satisfy. These interfaces, representing the flow of user interaction, are combined with the related processes implemented in the underlying system.

\section{Reservation}
\label{ch:Design:sec:Reservation}

As a fundamental unit, the reservation entity and its constituents serve as the foundation for the design process.  
Hence, multiple sources, such as the \acrshort{ocpp} standard, which is employed by most \acrshortpl{cs}, and the associated research in chapter \ref{ch:Literature Review}, are given primary consideration.
To cover the basic functionality for interacting with the underlying charging infrastructure, the reservation must contain specific properties that enable an unambiguous mapping between the entities provided in the standard solutions and those internal to the system.
Considering the absence of features not covered by existing standards, this thesis introduces several properties as a dedicated extension to the existing work, which should provide further possibilities in the management of reservations and related infrastructure.
Starting with the requirements for compliance with version 1.6 of the \acrshort{ocpp} standard, the selection inside table \ref{tab:reservation-ocpp-properties} includes the same essential information as the aforementioned protocol.

\begingroup
\setlength{\tabcolsep}{10pt} % Default value: 6pt
\renewcommand{\arraystretch}{1.5} % Default value: 1
\begin{table}[h]
    \centering
    \caption{Reservation properties based on 'Reserve Now' operation in \cite{noauthor_ocpp_nodate}.}
    \begin{tabular}{c|m{10cm}}
        Property & Description \\ \hline
        \acrshort{id} & Distinct identifier for the reservation within the \acrshort{cs} and the system \\
        Charging Station \acrshort{id} & Association to the corresponding \acrshort{cs} \\
        Connector \acrshort{id} & Association with the relevant reserved connector \\
        Tag \acrshort{id} & Association with the \acrshort{rfid} tag for the user \\
        Parent Tag \acrshort{id} & Association with the superior \acrshort{rfid} tag \\
        Expiry Date & Date when the reservation expires
    \end{tabular}
    \label{tab:reservation-ocpp-properties}
\end{table}
\endgroup

\noindent Based on these standard properties required for an \acrshort{ocpp} reservation, the author proposes the extensions listed in table \ref{tab:reservation-extended-properties}.

\begingroup
\setlength{\tabcolsep}{10pt} % Default value: 6pt
\renewcommand{\arraystretch}{1.5} % Default value: 1
\begin{table}[h]
    \centering
    \caption{Reservation properties extending \acrshort{ocpp} protocol in version 1.6.}
    \begin{tabular}{c|m{10cm}}
        Property & Description \\ \hline
        From Date & Date the reservation begins \\ 
        To Date & Date the reservation ends \\
        Arrival Time & Dedicated time the reservation starts \\
        Departure Time & Dedicated time the reservation ends \\
        Status & Reservation status according to \ref{ch:Design:sec:Reservation:ssec:Reservation Status} \\
        Type & Type of reservation according to \ref{ch:Design:sec:Reservation:ssec:Reservation Types} \\
        Car \acrshort{id} & Association to users car 
    \end{tabular}
    \label{tab:reservation-extended-properties}
\end{table}
\endgroup

\noindent Especially these enhancements should enable the system user to make an advance reservation, in addition to the already supported immediate reservations for locking connectors directly, by booking a single or a recurring reservation(s) within a selected date range and time slot.
Furthermore, the option to add the car to the booking should enable the compatibility of the reservation system for the integration of \acrshort{sg} applications, e.g. providing \acrshort{v2g} operations.
The introduction of the status and type properties serves as markers to distinguish between reservation types and the current status of the reservation. These details will be discussed in detail in the following subsections.
Properties such as prioritising reservations for different categories of users or integrating a pricing scheme are not covered by this work and could be developed as part of future work to extend this reservation system approach.
To impart a full understanding of the novel relationships between entities within the system and their use in the reservation framework, the following subsection outlines these newly constructed associations.

\subsection{Entity Relationships}
\label{ch:Design:sec:Reservation:ssec:Entity Relationships}

In order to organise the properties and entities within the reservation module and the rest of the system, it is necessary to examine the relationships that the specific entities, such as the \acrshort{cs} or a \acrshort{rfid} tag, have with a reservation and the quantity in which they can be used.
For presenting this type of mapping concerning the relationships, they are depicted as an entity relationship diagram to show the respective connections and to illustrate them in the figure \ref{fig:entity-relationship-diagram}.
Therefore, it is intended that the reservation can only reserve a particular \acrshort{cs} and connector during a given time. Additionally, either an individual user, identified by their associated \acrshort{rfid} tag or a group of users linked to a parental \acrshort{rfid} tag that represents this subset, may use the reservation. The chosen vehicle must also adhere to these restrictions.

\begin{figure}[h]
    \centering
    \includegraphics[scale=0.4]{resources/images/main/5_design/Entities.png}
    \caption{Entities and their relationships with the corresponding quantities.}
    \label{fig:entity-relationship-diagram}
\end{figure}

\noindent Beyond the relationships introduced during this design process, the following subsections provide a more detailed look at the newly introduced properties in terms of their relevance to later implementations and the functionality they provide.

\subsection{Reservation Status}
\label{ch:Design:sec:Reservation:ssec:Reservation Status}

According to the findings of Flocea et al. \cite{flocea_electric_2022}, the reservation process goes through several stages before it is completed. These stages can be depicted as a status, which indicates where the reservation is currently located in the reservation process and provides an overview of the options available to the relevant user or the system itself to interact with the reservation.
Therefore, the study mentioned above introduced four dedicated states, distinguishing between \textit{New}, \textit{Done}, \textit{In progress} and \textit{Cancelled}.
Considering these pre-existing states, this thesis proposes an extension that allows a more granular treatment of a reservation during its life cycle. Besides breaking down the particular steps within the life of a reservation, these additional states allow further functionality in terms of mitigating undesired behaviour and enforcing software-based rules in terms of the design parameter \textit{Enforceability} mentioned later.

\begingroup
\setlength{\tabcolsep}{10pt} % Default value: 6pt
\renewcommand{\arraystretch}{1.5} % Default value: 1
\begin{table}[h]
    \centering
    \caption{Introduced reservation states to extend the reservation life cycle.}
    \begin{tabular}{c|c|m{7cm}}
        Reservation Status & Equivalent in \cite{flocea_electric_2022} & Description \\ \hline
        Done & - & \acrshort{evu} was present and charged the \acrshort{ev} \\
        Scheduled & New & Planned reservation whose start date has not yet been reached \\
        In Progress & In progress & Reservation currently in progress \\
        Cancelled & Cancelled & The reservation has been cancelled \\
        Expired & - & The reservation has reached its expiration date. \\
        Unmet & - & The \acrshort{evu} did not arrive punctually
    \end{tabular}
    \label{tab:reservation-states}
\end{table}
\endgroup

\noindent Taking into account the current states, the proposed extensions additionally cover the scenarios of a user not arriving on time, the reservation itself expiring, the reservation being successfully fulfilled, and partially the scheduled state until the start of the reserved time slot. 
These aspects are not considered by Forcea et al. and should be integrated for addressing certain situations.
To characterise these particular situations, the following state diagram, shown in figure \ref{fig:reservation-states}, provides an overview of the possible transitions according to the conditions that must be met.

\begin{figure}[h]
    \centering
    \includegraphics[scale=0.4]{resources/images/main/5_design/ReservationStatusStates.png}
    \caption{State diagram with the corresponding state transitions for the possible reservation states within the implementation.}
    \label{fig:reservation-states}
\end{figure}

\noindent Not every reservation undergoes the same stages as depicted above due to its specific type. To provide an overview of each reservation type included in the reservation system design, the following subsection offers guidance on the potential variations.

\newpage

\subsection{Reservation Types}
\label{ch:Design:sec:Reservation:ssec:Reservation Types}

As outlined by Basmadjian and colleagues, a reservation system can accommodate various types of bookings. In addition to the immediate blocking of resources through the \textit{Ad-Hoc} reservation mechanism, the \textit{Planned} reservation allows for advance booking, opening up a wider range of possibilities for incorporating this functionality into the management of charging infrastructure \cite{basmadjian_interoperable_2019,basmadjian_reference_2020}.
Therefore, the authors distinguish between reservation types based on their \textit{enforceability} and propose uncertain and guaranteed reservations, which can be combined to form four possible variations.
The \textit{uncertain} reservations cannot guarantee that the reserved \acrshort{cs} and its corresponding parking spot will not be occupied by another vehicle upon arrival, whereas the \textit{guaranteed} reservation ensures this possibility.
Enabling this reliability requires the implementation of additional functionalities, such as the ability to control the physical infrastructure itself.
In this design process, considering no direct access to physical infrastructure services as blockers for parking and similar technologies, the resulting system tries to provide a hybrid reservation type combining the best from the world of \textit{uncertain} and \textit{guaranteed} reservations using only software-based solutions to provide the desired level of \textit{enforceability}.
Building a connection to the types found in the existing literature, the table \ref{tab:reservation-types} then maps the types used in this design and the subsequent implementation to the existing ones.

\begingroup
\setlength{\tabcolsep}{10pt} % Default value: 6pt
\renewcommand{\arraystretch}{1.5} % Default value: 1
\begin{table}[h]
    \centering
    \caption{Pre-defined reservation types and their mapping to the types used in this work.}
    \begin{tabular}{c|c|m{6.5cm}}
        Reservation Type & Equivalent in \cite{basmadjian_interoperable_2019,basmadjian_reference_2020} & Description \\ \hline
        Reserve Now & Ad-Hoc &Reserving the \acrshort{cs} and its corresponding connector immediately until a specified expiration date is reached. During this time, no other user can use the same connector for charging purposes. \\
        Planned Reservation & Planned & Reserves the \acrshort{cs} and the corresponding connector in advance, enabling recurring reservations by selecting a date range. The connector and the \acrshort{cs} will stay available until the reservation starts.
    \end{tabular}
    \label{tab:reservation-types}
\end{table}
\endgroup

After the introduction of the reservation entity, the following section deals with the design of the reservation system, together with the considered processes and the system architecture. 

\newpage

\section{Reservation System}
\label{ch:Design:sec:Reservation System}

To empower the user to create and manage their reservations, a reservation system is required. This section proposes a design for the implementation of the system that will be the subject of the following chapter. Firstly, the necessary capabilities considered relevant in this work are discussed, as well as the needed design criteria in terms of the system and its architecture.

\subsection{Design Criteria}
\label{ch:Design:sec:Reservation System:ssec:Design Criteria}

Establishing a functional reservation system for managing charging infrastructure entails the consideration of several design parameters, as outlined by Basmadjian et al.\cite{basmadjian_reference_2020}. Apart from \textit{Enforceability}, \textit{Planning}, \textit{Fees}, aspects such as \textit{Data Availability}, \textit{Roaming} and \textit{Scheduling} are important design parameters that a system design has to address.
Regarding the given options and possibilities present in this design approach, the author of this work opts for the inclusion of \textit{Enforceability} alongside \textit{Planning}, \textit{Data Availability} and \textit{Scheduling}. 
Furthermore, the \acrshort{fcfs} approach has been selected as the most appropriate method for mapping real-world charging session bookings.\\
\noindent In accordance with the terminology employed by \cite{basmadjian_reference_2020}, the system can be defined as follows:

\begin{eqnarray*}
\scriptstyle \biggl\{ "Limited"_{Enforceability}, "Yes"_{Planning}, "No"_{Fee}, "Yes"_{Data\ Availability}, "No"_{Roaming}, "FCFS"_{Scheduling} \biggr\}
\end{eqnarray*}

\noindent To achieve these objectives, the following subsections outline the relevant capabilities that meet these design criteria.

\subsection{Management Capabilities}
\label{ch:Design:sec:Reservation System:ssec:Management Capabilities}

When managing various entities, the most commonly utilized functionalities involve creating, updating, or deleting related information. Alongside these processes, the acronym CRUD (create, read, update, delete) is often referenced. In reservation systems, the context permits an additional operation known as cancellation, which invalidates the selected reservation.
As outlined in chapter \ref{ch:Requirements Engineering}, the relevant use cases for the previously mentioned functions are already identified. In addition to these general features, the subsection also explains the use case for enabling them on demand.
Considering these capabilities, the following management capabilities cover at least parts of the design choices such as \textit{Enforceability}, \textit{Planning}, \textit{Data Availability} and \textit{Scheduling}, alongside the scheduling capabilities described in subsection \ref{ch:Design:sec:Reservation System:ssec:Scheduling Capabilities}.

\subsubsection{Create Reservation}
\label{ch:Design:sec:Reservation System:ssec:Management Capabilities:sssec:Create Reservation}

In order to enable all of the subsequent procedures, reservation creation is a mandatory function, which can be considered the initial process. The user inputs their information through the chosen application, followed by system-side validation that leads to further processing in subsequent steps or immediate termination of the process.
Afterwards, the system verifies for any existing reservations with the same identifier to avoid duplicates and ensure consistency with respect to the underlying database schema. If a reservation with the same identifier exists, the process tests for the same user according to their \acrshort{rfid} tag and updates the reservation according to matching tags or rejects it to prevent privilege escalation.
Next, the system checks for other reservations in the defined time range and ensures that there are no collisions, also known as overlaps, between individual reservations according to their dates and time slots.
Any collisions detected, as well as the escalation of privileges during the update process of an existing reservation, result in a predetermined end.
If no conflict is detected, the reservation type is checked and, in the case of a \textit{Reserve Now} reservation, the encapsulated 'Reserve Now' operation defined in the \acrshort{ocpp} standard \cite{noauthor_ocpp_nodate} is triggered. This operation prompts the system to send a request to the relevant \acrshort{cs} to immediately lock the connector. 
Otherwise, when dealing with a \textit{Planned Reservation}, the selected time range will be checked and if the start time is approaching, similar to the case of a \textit{Reserve Now} reservation, the appropriate \acrshort{cs} is contacted.
Finally, the reservation is saved and a notification, described in more detail in subsection \ref{ch:Design:sec:Reservation System:ssec:Notification Capabilities}, is sent to the user to inform him or her of the successful creation.

\begin{figure}[h]
    \centering
    \includegraphics[scale=0.4]{resources/images/main/5_design/processes/ReservationCreate.png}
    \caption{Process flow with all according steps to create a reservation.}
    \label{fig:create-reservation-flowchart}
\end{figure}

\noindent To handle the essential inputs, the subsequent mockups offer a suitable user interface encompassing all the required information provided by the user. Therefore, a proposition for the \acrshort{gui} of both the web and mobile app is presented.

\begin{figure}[h]
    \centering
    \begin{subfigure}[c]{0.6\textwidth}
        \includegraphics[width=\textwidth]{resources/images/main/5_design/mockups/create_reservation/web/Reservation_Create.png}
        \captionsetup{skip=43pt}
        \caption{Create a reservation in the web application.}
        \label{fig:web-create-reservation-mockup}
    \end{subfigure}
    \hfill
    \begin{subfigure}[c]{0.3\textwidth}
        \includegraphics[width=\textwidth,height=1.6\textwidth,keepaspectratio]{resources/images/main/5_design/mockups/create_reservation/mobile/Create_Reservation.png}
        \caption{Create a reservation in the mobile application.}
        \label{fig:mobile-create-reservation}
    \end{subfigure}
    \caption{Mockups for the user interface of the mobile and web application relating to the creation of reservations.}
    \label{fig:mockups-create-reservation}
\end{figure}

\newpage

\subsubsection{Update Reservation}
\label{ch:Design:sec:Reservation System:ssec:Management Capabilities:sssec:Update Reservation}

To modify and manage existing reservations, the user is able to update a reservation. In addition to changing the date, time and \acrshort{cs}, modifications can be made to the connectors as well as the provided vehicle, if available. 
Similar to the creation process, the corresponding user input and changes are validated, resulting in further processing of the update process or immediate termination.
Regarding successful validation, the system searches for a reservation with the same \acrshort{id} to update it accordingly.
To capture potential events, such as unsaved reservations, the update process turns into the create process as described above to mitigate the erroneous behavior.
Otherwise, the reservation properties received are updated. If another time or \acrshort{cs} is selected, the reservation on the previous station is cancelled, and the newly selected \acrshort{cs} and connector are reserved.
Afterwards, a series of steps similar to the creation process described above take place. This includes checking for new collisions created by other users during the update and rejecting them accordingly, preparing the notifications for the selected \acrshort{cs} and finally saving the updated reservation.
This also contains another notification that the user will receive upon successful completion.

\begin{figure}[h]
    \centering
    \includegraphics[scale=0.4]{resources/images/main/5_design/processes/ReservationUpdate.png}
    \caption{Process flow with all according steps to update a reservation.}
    \label{fig:update-reservation-flowchart}
\end{figure}

\noindent Due to a comparable range of input, the recommendations for the respective interfaces of the mobile and web applications are almost identical, as shown in the figure \ref{fig:mockups-update-reservation} below.

\begin{figure}[h]
    \centering
    \begin{subfigure}[c]{0.6\textwidth}
        \includegraphics[width=\textwidth]{resources/images/main/5_design/mockups/update_reservation/web/Edit_Reservation.png}
        \captionsetup{skip=33pt}
        \caption{Update a reservation in the web application.}
        \label{fig:web-update-reservation-mockup}
    \end{subfigure}
     \hfill
     \begin{subfigure}[c]{0.3\textwidth}
         \includegraphics[width=\textwidth,height=1.6\textwidth,keepaspectratio]{resources/images/main/5_design/mockups/update_reservation/mobile/Edit_Reservation.png}
         \caption{Update a reservation in the mobile application.}
         \label{fig:mobile-update-reservation-mockup}
    \end{subfigure}
    \caption{Mockups for the user interface of the mobile and web application concerning the update of reservations.}
    \label{fig:mockups-update-reservation}
\end{figure}

\clearpage

\subsubsection{Cancel Reservation}
\label{ch:Design:sec:Reservation System:ssec:Management Capabilities:sssec:Cancel Reservation}

In addition to making a reservation, the user needs the option to cancel it in the case of changes due to unforeseen or unexpected circumstances.
Therefore, this cancellation operation encapsulates the predefined 'Cancel Reservation' feature in \acrshort{ocpp} version 1.6 \cite{noauthor_ocpp_nodate}. As well as cancelling the active reservation on the \acrshort{cs}, it also allows cancelling  scheduled reservations, taking into account the reservation life cycle as described in subsection \ref{ch:Design:sec:Reservation:ssec:Reservation Status}.
Initially, the designed process checks for an existing reservation with the same identifier in the database. If the reservation does not exist, the execution ends. This practice avoids the processing of unnecessary information and prevents communication with the corresponding customer service.
If the reservation exists, the reservation status is validated to cancel only reservations according to the predefined state transitions in figure \ref{fig:reservation-states}. Furthermore, the current status of the reservation is determined, which results in a 'Cancel Reservation' request contacting the appropriate \acrshort{cs} if the reservation is in progress.
After the reservation status is set to \textit{Cancelled}, the reservation is saved once more, and a user notification containing the reservation's status transition is sent to the user.

\begin{figure}[h]
    \centering
    \includegraphics[scale=0.4]{resources/images/main/5_design/processes/ReservationCancel.png}
    \caption{Process flow with all according steps to cancel a reservation.}
    \label{fig:cancel-reservation-flowchart}
\end{figure}

\noindent Regarding the atomic nature of this function, it is designed as a dedicated button within the mockups presented below. To confirm the cancellation of a reservation and complete the process, users will be presented with a confirmation dialogue in both applications, which is not displayed in the illustrations below.

\begin{figure}[h]
    \centering
     \begin{subfigure}[c]{0.6\textwidth}
         \includegraphics[width=\textwidth]{resources/images/main/5_design/mockups/cancel_reservation/web/Cancel_Reservation.png}
         \captionsetup{skip=33pt}
         \caption{Cancel a reservation in the web application.}
         \label{fig:web-cancel-reservation-mockup}
    \end{subfigure}
     \hfill
     \begin{subfigure}[c]{0.3\textwidth}
         \includegraphics[width=\textwidth,height=1.6\textwidth,keepaspectratio]{resources/images/main/5_design/mockups/cancel_reservation/mobile/Cancel_Reservation.png}
         \caption{Cancel a reservation in the mobile application.}
         \label{fig:mobile-cancel-reservation-mockup}
    \end{subfigure}
    \caption{Mockups for the user interface of the mobile and web application regarding the cancellation of reservations.}
    \label{fig:mockups-cancel-reservation}
\end{figure}

\newpage

\subsubsection{Delete Reservation}
\label{ch:Design:sec:Reservation System:ssec:Management Capabilities:sssec:Delete Reservation}

Keeping in mind the management capabilities and complying with Article 17 of the \acrfull{gdpr}, which outlines the 'Right to Erasure' \cite{noauthor_art_2018}, the system offers the functionality to delete a reservation.
Therefore, this process functions similarly to the cancellation process described above, which also implies the design of the corresponding mockups.
Initially, the system storage is checked for the existence of the reservation identifier, which results in process termination if the reservation is not found. 
Otherwise, the reservation is queried and the reservation status is checked. If the status of the reservation is \textit{In Progress}, it is immediately cancelled. This is followed by the permanent deletion from the database. 

\begin{figure}[h]
    \centering
    \includegraphics[scale=0.4]{resources/images/main/5_design/processes/ReservationDelete.png}
    \caption{Process flow with all according steps to delete a reservation.}
    \label{fig:delete-reservation-flowchart}
\end{figure}

\noindent As mentioned previously, no proposals have been made for the corresponding user interfaces. The author suggests that the deletion feature could be implemented in a similar manner to the 'Cancel Reservation' method discussed earlier. Therefore, an extra design approach for the application interfaces is not part of this work.

\newpage

\subsubsection{Enable Reservations}
\label{ch:Design:sec:Reservation System:ssec:Management Capabilities:sssec:Enable Reservations}

Considering the reservation system as an independent part of the whole system, the user should have the possibility to activate or deactivate the respective functionality according to his needs.
Therefore, this feature is represented as a switch for controlling the functionality of the reservation system along with all its procedures. Similar to the approach presented in \cite{orcioni_ev_2020}, which describes the reservation system as a dedicated service. According to this proposal, it is possible to deactivate it as a module integrated as a subsystem.
This allows such functionality to be implemented alongside the processes described in existing standards, such as \acrshort{ocpp} or \acrshort{ocpi}, and does not interfere with their implementations.
Like the 'Delete Reservation' function, no user interface mockups are provided due to the simple nature of the function. In the subsequent implementation, this toggle should be considered within a central management interface that configures the functionalities of the different tenants. 

\newpage

\subsection{Scheduling Capabilities}
\label{ch:Design:sec:Reservation System:ssec:Scheduling Capabilities}

To address the scheduling capabilities discussed in subsection \ref{ch:Design:sec:Reservation System:ssec:Design Criteria}, the subsequent procedures have been developed to manage reservations independently.
This should enable the system to maintain a consistent state and autonomously manage the associated charging infrastructure.
Therefore, the scheduling, expiry, and release procedures are created to handle the reservations according to their current state.

\subsubsection{Schedule Reservation}
\label{ch:Design:sec:Reservation System:ssec:Scheduling Capabilities:sssec:Schedule Reservation}

By utilising the scheduling process, the system identifies present and forthcoming reservations from the database according to a pre-defined threshold. 
If there are no upcoming reservations in the near future, the scheduler process is terminated, otherwise the \acrshortpl{cs} and connectors defined in each reservation are reserved.
Similar to the create and update functions described within the management capabilities subsection \ref{ch:Design:sec:Reservation System:ssec:Management Capabilities}, the \acrshort{ocpp} 'Reserve Now' operation \cite{noauthor_ocpp_nodate} is implemented to provide standards-compliant communication between the reservation system and the \acrshort{cs}.
If connectors are in an occupied state, the 'Stop Transaction' operation is performed, resulting in a immediate termination of the current charging session and returning the \acrshort{cs} to an available state.
Allowing the system to carry out the upcoming reservation on the released connector, resulting in the \acrshort{cs} accepting the 'Reserve Now' request.
Consequently, the reservation status is updated to 'In Progress' and saved once again.

\begin{figure}[h]
    \centering
    \includegraphics[scale=0.4]{resources/images/main/5_design/processes/scheduler/SynchronizeReservation.png}
    \caption{Process for scheduling upcoming reservation on a specified \acrshortpl{cs}.}
    \label{fig:schedule-reservation-flowchart}
\end{figure}

\noindent Besides providing autonomous functionality for synchronising reservations with the corresponding \acrshortpl{cs}, this procedure allows the execution of recurring reservations during the defined period. 
Furthermore, if the user disconnects the connector during the reservation period, the underlying logic ensures that the reservation remains active until a certain threshold is reached.

\subsubsection{Expire Reservation}
\label{ch:Design:sec:Reservation System:ssec:Scheduling Capabilities:sssec:Expire Reservation}

Forming the counterpart to the scheduling process, which synchronises upcoming and ongoing reservations with the \acrshortpl{cs}, the expiring process handles all reservations that are neither cancelled nor expired and have reached their expiration date.
The process ends immediately upon discovering any such reservations. Otherwise, the status of the reservation is changed to 'Expired', the user is notified and the reservation is saved again.

\begin{figure}[h]
    \centering
    \includegraphics[scale=0.4]{resources/images/main/5_design/processes/scheduler/UpdateExpiredReservations.png}
    \caption{Process flow for reservations reaching their expiration date.}
    \label{fig:expire-reservation-flowchart}
\end{figure}

\noindent In cooperation with the 'Free Reserved Connector' procedure described in the sub-subsection \ref{ch:Design:sec:Reservation System:ssec:Scheduling Capabilities:sssec:Free Reserved Connector}, the 'Expire Reservation' procedure handles the transition from the active state to a final state that invalidates the reservation.
Using the 'Expired' status handles reservations that are not correctly updated according to the 'Done', 'Cancelled' or 'Unmet' states, caused by information or connection loss between the management instance and the \acrshort{cs}.
This approach ensures that concurrent active reservations are minimised and provides a sort of self-healing process to clean up the growing data store.

\subsubsection{Free Reserved Connector}
\label{ch:Design:sec:Reservation System:ssec:Scheduling Capabilities:sssec:Free Reserved Connector}

In the case of a reserved connector being blocked, the respective user who made the booking did not show up and failed to cancel the reservation, thus preventing the connector from being available again.
To reduce the likelihood of this situation, the system provides a process to automatically remove unmet reservations from the \acrshortpl{cs} and associated connectors.
Therefore, the system selects reservations that are already in progress, the connector is not in the 'Charging' state and the specified arrival time is overdue by a certain threshold.
After identifying reservations that meet these criteria, the system cancels them on the \acrshortpl{cs}. Once the station successfully acknowledges the cancellation, the reservation status is updated to 'Unmet'.
Otherwise, the process ends immediately. The same rule applies if the \acrshort{cs} does not permit the cancellation of the reservation.

\begin{figure}[h]
    \centering
    \includegraphics[scale=0.4]{resources/images/main/5_design/processes/scheduler/CancelUnmetReservation.png}
    \caption{Process flow for releasing connectors with unfulfilled reservations.}
    \label{fig:free-connector-flowchart}
\end{figure}

\newpage

\subsection{Notification Capabilities}
\label{ch:Design:sec:Reservation System:ssec:Notification Capabilities}

Beside the management and automation techniques presented above, the exchange of information with the user encompasses one aspect of the predetermined design parameters.
Following the methods in \cite{zarkeshev_charging_2018}, the reservation system should interact with the user in a certain way to inform him or her about upcoming reservations or schedule changes, which the user should consider for further planning activities.
Therefore, this system design outlines several scenarios that are considered by the author to be relevant for the reservation system to inform the user according to the changes made to the reservations. 
This also satisfies the aspects of the \textit{Data Availability} design parameter in a more general sense. Regarding the aforementioned work, this parameter is limited solely to data regarding the status of charging stations and connectors. However, additional expansions could be made to include data on reservation statuses.
For a clearer insight into the constraints associated with each notification type, a brief explanation of the corresponding scenario follows. 

\subsubsection{Reservation Status Changed}
\label{ch:Design:sec:Reservation System:ssec:Notification Capabilities:sssec:Reservation Status Changed}

Besides the users' ability to manually configure their reservations, the bulk of the work is performed by background tasks that run in parallel to the main processes.
For users, it is therefore neither useful nor possible to constantly monitor the current status of their reservations. Therefore, it is necessary to send notices containing information about changes in the status of a reservation throughout the life cycle of the latter.
Due to its supportive role within the system context, this type of functionality could be integrated into one of the previously designed processes. 

\noindent Considering an intended integration, the following parts of the design processes could be proposed as suitable extension points using this notification capability:
\begin{itemize}
    \item \textbf{Schedule Reservation}
    \item \textbf{Expire Reservation}
    \item \textbf{Free Reserved Connector}
\end{itemize}

\subsubsection{Reservation Upcoming}
\label{ch:Design:sec:Reservation System:ssec:Notification Capabilities:sssec:Reservation Upcoming}

In order to address the emergence of reservations, two distinct types of notifications can be extracted from the more generic one. One is the notification of the user to whom the reservation belongs and the other is the notification of the user who intends to charge at a \acrshort{cs} a reservation comes up in the near future. 
Both scenarios are represented by the notification type of a 'Reservation Upcoming'.

\begin{description}
    \item[Upcoming Reservation Notification] According to the process of scheduling reservations, users must receive a notification if their reservation is due to start in the near future. This type of notification should provide the user with the relevant metadata about the reservation and its commencement, ensuring they are informed about the upcoming reservation.
    \item[Upcoming Reservation Warning] Message that the charging station that the driver is currently using has a reservation in the pipeline that conflicts with the estimated duration of the charging session. Therefore, the driver should consider relocating the car. Otherwise, the charging session is stopped when the reservation starts, and the \acrshort{ev} is no longer charged, making it pointless to occupy this station.
\end{description}

\noindent Considering the intended use of this notification in the above designs, the following parts can be identified:
\begin{itemize}
    \item \textbf{Schedule Reservation}
\end{itemize}

\subsubsection{Charging Station Blocked}
\label{ch:Design:sec:Reservation System:ssec:Notification Capabilities:sssec:Charging Station Blocked}

Considering the 'Upcoming Reservation Warning' mentioned previously, only the user occupying the reserved \acrshort{cs} is notified that it is potentially blocking a reserved port. To complement this scenario, the user the upcoming reservation belongs to requires a notification as well.
Therefore, the 'Charging Station Blocked' notification ensures that the user is informed that unintended behaviour occurred. In this way, if the \acrshort{cs} is still blocked and the other \acrshort{evu} does not remove the car from the parking lot accordingly, the notified user is able to switch the reservation to another available \acrshort{cs} in that area.
Despite stopping the charging session in accordance with the 'Schedule Reservation' process, the car could still end up blocking the car bay for the arriving user, resulting in a negative user experience. Therefore, the system is intended to provide such mitigation techniques to inform the user in advance and allow him or her to adjust the reservation.

\noindent In view of the intended use, the following parts of the design processes are identified for the intended use of this message type:
\begin{itemize}
    \item \textbf{Schedule Reservation}
\end{itemize}

\subsubsection{Reservation Cancelled}
\label{ch:Design:sec:Reservation System:ssec:Notification Capabilities:sssec:Reservation Cancelled}

When considering the 'Reservation Status Changed' notification introduced earlier, this notification type seems to be a duplicate of a still existing function.
However, there should be a specific notification in the system design to inform the user when a reservation is being cancelled. The system's implementation and internal administrative structure does not allow only the user who created the reservation to cancel it. 
If an administrator performs this operation, the user should also be notified that their reservation is about to be cancelled. 
Therefore, this notification presents an extension on top of the changing status notification to provide this information.

\noindent Based on the intended use, the following components of the design work are identified for the intended use of this notification:
\begin{itemize}
    \item \textbf{Cancel Reservation}
\end{itemize}

\subsubsection{Reservation Unmet}
\label{ch:Design:sec:Reservation System:ssec:Notification Capabilities:sssec:Reservation Unmet}

Based on the same idea as the 'Reservation Cancelled' message, this kind of notification enables the system to inform the user about unwanted behavior.
When a user fails to arrive within the designated time frame for their reservation, the 'Free Reserved Connector' process usually  takes care of these reservations. 
However, given the damaging effects of such a behaviour, like blocking a charging session that could be used by another user, and the associated loss of profit, when combined with the charges for the reservation.
To reduce the occurrence of such actions, the system should send a message advising the user that their current reservation is no longer viable according to this process. Furthermore, it should be accompanied by a warning to remind the user to cancel a reservation if he or she no longer needs it, in order to cultivate more altruistic behavior.

\noindent Regarding its particular purpose, the following processes could be identified for the integration of this notification type:
\begin{itemize}
    \item \textbf{Free Reserved Connector}
\end{itemize}

\noindent The integration possibilities and application domains addressed in this chapter are derived from the current state of the standards used in this thesis and are based on the considerations and design principles available at this stage of the design work. 
As these functionalities and features are evolving, their use could be extended to other areas and related systems.

%% LaTeX2e class for student theses
%% sections/main/6_implementation.tex
%%
%% Karlsruhe University of Applied Sciences
%% Faculty of  Computer Science and Business Information Systems
%%
%% --------------------------------------------------------
%% | Derived from sdqthesis by Erik Burger burger@kit.edu |
%% --------------------------------------------------------

\chapter{Implementation}
\label{ch:Implementation}

\section{Resources}
\label{ch:Design:sec:Resources}

In case of a reservation system, the basic resource this system has to handle is defined as a \verb|reservation|. The according  endpoints exposed by the application for providing the functionalities to interact with the resource and required in the frontend parts of the application are listed below.

\begingroup
\setlength{\tabcolsep}{10pt} % Default value: 6pt
\renewcommand{\arraystretch}{1.5} % Default value: 1
\begin{table}[!ht]
\centering
\caption{Provided \acrshort{rest} endpoints }
    \begin{tabular}{l|c|m{5cm}}
    Resource Identifier & HTTP Method & Functionality \\ \hline
    \multicolumn{3}{c}{\verb"/reservations"} \\ \hline
    \verb|/| & \verb|GET|, \verb|POST|, \verb|DELETE| & Get a list of reservations \\
    \verb|/{id}| & \verb|GET|, \verb|PUT|, \verb|DELETE| & Read, update and delete operation for a specific reservation \\
    \verb|/{id}/cancel| & \verb|PUT| & Cancel a existing reservation by its ID \\
    \verb|/action/export| & \verb|GET| & Export existing reservations \\
    \multicolumn{3}{c}{\verb"/charging-stations"} \\ \hline
    \verb|/{id}/reserve/now| & \verb|PUT| & Reserve a connector at a \acrshort{cs} with ID \\
    \verb|/{id}/reservation/cancel| & \verb|PUT| & Cancel a reservation on a connector at a \acrshort{cs} with ID \\
    \verb|/reservation/availability| & \verb|GET| & Get reservable \acrshortpl{cs} for a specific time range 
    \end{tabular}
\label{tab:rest-endpoints}
\end{table}
\endgroup

\section{Implemented Use Cases}
\label{ch:Implementation:sec:Implemented Use Cases}

Afterwards, the implemented use cases will be described in details.

\subsection{Create Reservation}
\label{ch:Implementation:sec:Implemented Use Cases:ssec:Create Reservation}

\begin{figure}[!ht]
    \centering
    \includegraphics[scale=0.4]{resources/images/main/6_implementation/processes/ReservationCreate.png}
    \caption{Flow of information through the single components of the backend service}
    \label{fig:create-reservation-seq-flow}
\end{figure}

\subsection{Update Reservation}
\label{ch:Implementation:sec:Implemented Use Cases:ssec:Update Reservation}


\begin{figure}[!ht]
    \centering
    \includegraphics[scale=0.4]{resources/images/main/6_implementation/processes/ReservationUpdate.png}
    \caption{Flow of information through the single components of the backend service}
    \label{fig:update-reservation-seq-flow}
\end{figure}

\subsection{Delete Reservation}
\label{ch:Implementation:sec:Implemented Use Cases:ssec:Delete Reservation}

\begin{figure}[!ht]
    \centering
    \includegraphics[scale=0.4]{resources/images/main/6_implementation/processes/ReservationDelete.png}
    \caption{Flow of information through the single components of the backend service}
    \label{fig:delete-reservation-seq-flow}
\end{figure}

\subsection{Cancel Reservation}
\label{ch:Implementation:sec:Implemented Use Cases:ssec:Cancel Reservation}


\begin{figure}[!ht]
    \centering
    \includegraphics[scale=0.4]{resources/images/main/6_implementation/processes/ReservationCancel.png}
    \caption{Flow of information through the single components of the backend service}
    \label{fig:cancel-reservation-seq-flow}
\end{figure}

\subsection{Schedule Reservation}
\label{ch:Implementation:sec:Implemented Use Cases:ssec:Schedule Reservation}

\dots

% \begin{figure}[!ht]
%     \centering
%     \includegraphics[scale=0.4]{resources/images/main/6_implementation/processes/scheduler/SynchronizeReservation.png}
%     \caption{Flow of information through the single components of the backend service}
%     \label{fig:schedule-reservation-flow}
% \end{figure}

\subsection{Expire Reservation}
\label{ch:Implementation:sec:Implemented Use Cases:ssec:Expire Reservation}

\dots

\subsection{Free reserved connectors}
\label{ch:Implementation:sec:Implemented Use Cases:ssec:Free reserved connectors}

\dots

% \begin{figure}[!ht]
%     \centering
%     \includegraphics[scale=0.4]{resources/images/main/6_implementation/processes/scheduler/CancelUnmetReservation.png}
%     \caption{Flow of information through the single components of the backend service}
%     \label{fig:free-connector-flow}
% \end{figure}
%% LaTeX2e class for student theses
%% sections/main/7_validation.tex
%%
%% Karlsruhe University of Applied Sciences
%% Faculty of Computer Science and Business Information Systems
%%
%% --------------------------------------------------------
%% | Derived from sdqthesis by Erik Burger burger@kit.edu |
%% --------------------------------------------------------

\chapter{Analysis and Validation}
\label{ch:Analysis and Validation}

After the design and the implementation of the proposed application model and its functionalities supposed to be necessary in the context of a reservation system for charging infrastructure management. This chapter concludes the assumptions made during the process of designing and implementing the aforementioned aspects.
This involves the assessment of the stated goals regarding the final system design and the degree of fulfillment.
Furthermore, open issues are discussed, which could not be addressed in the context of this work but considered as necessary for compulsory for a sufficient experience using this particular kind of system as well as its limitations.

\section{Achievement of Objectives}
\label{ch:Analysis and Validation:sec:Achievement of Objectives}

For assessing the produced output, this section recaps the relevant objectives stated at the start of this particular work and set them into the context of the achieved targets.
Beginning with a recap of the formerly state objectives introduced in section \ref{ch:Requirements Engineering:sec:Goals} as part of the design chapter.
%% --------------------------------------------------------
%% Goal 1 - Management Capabilities
%% --------------------------------------------------------
Introduced as first goal, the capability of management concerning the reservations created and administered within the system has top priority. According the designed processes, the ability to create, edit, cancel and to delete a reservation entity is identified as target.
Furthermore, the configuration of the system to provide the extended reservation capability in contrast to the standard operations defined in \acrshort{ocpp} are considered as relevant.
%% --------------------------------------------------------
%% Goal 2 - Self-Healing and Autonomous Processes
%% --------------------------------------------------------
Self-healing and autonomy regarding the management of the contained entities and the underlying infrastructure requiring as less interaction with a physical user, the second goal sets the objective to provide capabilities which allow the system to fully manage the life-cycle of its concerning entities on its own.

%% --------------------------------------------------------
%% Goal 3 - Support of relevant standards
%% --------------------------------------------------------
Regarding the previously mentioned \acrshort{ocpp} standard primarily used throughout this work, 
%% --------------------------------------------------------
%% Goal 4 - Modular Design
%% --------------------------------------------------------

Considering the actual output and the deficiencies, the following ratings could be assigned to the goals previously recapped.
%% --------------------------------------------------------
%% Goal 1 - Management Capabilities
%% --------------------------------------------------------

%% --------------------------------------------------------
%% Goal 2 - Self-Healing and Autonomous Processes
%% --------------------------------------------------------

%% --------------------------------------------------------
%% Goal 3 - Support of relevant standards
%% --------------------------------------------------------

%% --------------------------------------------------------
%% Goal 4 - Modular Design
%% --------------------------------------------------------


\section{Open Issues}
\label{ch:Analysis and Validation:sec:Open Issues}

Afterwards, considered use cases for implementation are listed, which could not be achieved during the processing time of this work, but as part of future implementations.

% Smart Charging Integration
% Re-scheduling of existing reservations and the according confirmation message for the user
% Fees and Payment
% Parent RFID Tag Integration
% Enable Reservations on CS via the UI not using the configuration metadata
% Reservations on logical connector 0

\section{Limitations}
\label{ch:Analysis and Validation:sec:Limitations}

According to the system as non-physical entity, it could not cover all aspects of managing the connected infrastructure. Beside the system itself, other controlling entities are necessary which could interact directly with the \acrshortpl{evu} and could take immediate action in certain cases.

% Guaranteeing the reservation considering no way to block the parking lot


% Conclusion 
%% LaTeX2e class for student theses
%% sections/conclusion.tex
%%
%% Karlsruhe University of Applied Sciences
%% Faculty of Computer Science and Business Information Systems
%%
%% --------------------------------------------------------
%% | Derived from sdqthesis by Erik Burger burger@kit.edu |
%% --------------------------------------------------------


\chapter{Resume and Outlook}
\label{ch:Resume and Outlook}

Regarding the results achieved as part of the previous chapters, the following sections provide a resume summarizing the essential parts of this work and providing an outlook considering the limitations of the elaborated approach and prepare possible problem statements for future work in this field.

\section{Resume}
\label{ch:Resume and Outlook:sec:Resume}

In retrospect, this work elaborates

\section{Outlook}
\label{ch:Resume and Outlook:sec:Outlook}

The usage of \acrshortpl{ev} and the requirement for an expanding charging infrastructure leads to necessity for further development in this area. Beside the basic functionalities provided today by different standards, a need for further development especially in the management of charging infrastructure is inevitable. 

\subsection{Future Work}
\label{ch:Resume and Outlook:sec:Outlook:ssec:Future Work}

Integration of Smart Charging service of the 'Open e-Mobility' solution as well as the simulation of the offered reservation approach utilizing tools like \href{https://github.com/aicenter/agentpolis}{AgentPolis}.
Furthermore, the notification regarding block vehicles on \acrshortpl{cs} parking lots could be integrated.
Additionally, optimization problems could be formulated based on the users preferences according the time, duration and charging power they require.
Reservations on mobile charging stations could be included as well.
As well as mapping to other existing open source standards to allow different kind of charging infrastructure to be registered.


%% --------------------
%% |   Bibliography   |
%% --------------------

%% Add entry to the table of contents for the bibliography
\printbibliography[heading=bibintoc]


%% ----------------
%% |   Appendix   |
%% ----------------
% \appendix
% %% LaTeX2e class for student theses
%% sections/apendix.tex
%%
%% Karlsruhe University of Applied Sciences
%% Faculty of  Computer Science and Business Information Systems
%%
%% --------------------------------------------------------
%% | Derived from sdqthesis by Erik Burger burger@kit.edu |
%% --------------------------------------------------------



\iflanguage{english}
{\chapter{Appendix}}    % english style
{\chapter{Anhang}}      % german style
\label{chap:appendix}


%% -------------------
%% | Example content |
%% -------------------
\section{First Appendix Section}
\label{sec:appendix:FirstSection}

\setcounter{figure}{0}

\dots
%% ---------------------
%% | / Example content |
%% ---------------------


\end{document}
