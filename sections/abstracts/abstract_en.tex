%% LaTeX2e class for student theses
%% sections/abstract_en.tex
%%
%% Karlsruhe University of Applied Sciences
%% Faculty of  Computer Science and Business Information Systems
%%
%% --------------------------------------------------------
%% | Derived from sdqthesis by Erik Burger burger@kit.edu |
%% --------------------------------------------------------


\Abstract
% Problem
Due to the continuously developing technologies in the context of electric mobility leading to an increased load on charging as well as the existing transport infrastructure, bottlenecks in the use of charging stations are inevitable.
Despite the application of intelligent mechanisms for load distribution and the use of computer-aided processes for charging electric vehicles, also known as \gls{smartcharging}, scenarios such as  management of charging infrastructure as part of reservations receive only little attention. 
Therefore, the need for a system-controlled approach for administration and distribution of charging facilities resulted as a key aspect for the further establishment of \acrfull{ev} within society and their acceptance as a safe and reliable means of transportation.\\
% Target
Which functional and non-functional requirements should such a system fulfill? This question is examined from different perspectives using detailed use cases embedded in a real-world scenario in this paper. 
The process of formulating and solving this particular kind of problems statement was an integral part of this master's thesis in cooperation with SAP SE.\\
% Approach
The fist part of this work started with theoretical considerations of already existing functionality covered by open industry standards, such as the \acrfull{ocpp} or the \acrfull{ocpi}. Because of their state as open standard, many manufacturers of electric charging stations implementing these features in their products by default. Hereinafter, the collection and analysis of requirements followed, which were not part of one of the already listed standards during the processing time of this thesis. Concluding, a subset of the identified functional modules were utilized to implement the necessary basic functionality in form of a \acrshort{poc}.\\
% Result
After the theoretical part isolating the necessary feature set for building a charging station reservation system, a basic set of operations and processes covering this functionality is implemented as part of the open source software \textbf{Open E-Mobility}, developed by SAP SE. Beside the implementation a successful simulation could be performed by utilizing the underlying scenario and the formulated use cases. 
The implementation of the reservation module of charging infrastructure could be integrated as an independent functionality within the different application components such as the web-based \href{https://github.com/JulianHBuecher/ev-dashboard}{user interface}, the \href{https://github.com/JulianHBuecher/ev-mobile}{mobile application}, as well as the \href{https://github.com/JulianHBuecher/ev-server}{backend service}. According to existing reservation systems, the resulting application demonstrates a process users could manage assigned charging infrastructure and reserve it for a certain period of time.\\
% Further Research Areas
According to the underlying implementation, reservations could be used within other scenarios in the described problem space. For example in the context of \gls{smartcharging}. In combination with the time periods of the individual reservations, approaches regarding bidirectional charging, better known as \acrfull{v2g}, for prevention of load peaks within the grid, are conceivable.