%% LaTeX2e class for student theses
%% sections/abstract_en.tex
%%
%% Karlsruhe University of Applied Sciences
%% Faculty of  Computer Science and Business Information Systems
%%
%% --------------------------------------------------------
%% | Derived from sdqthesis by Erik Burger burger@kit.edu |
%% --------------------------------------------------------


\Abstract
% Problem
Due to the continuous development of technologies in the context of electric mobility leading to an increased load on charging, as well as the existing transport infrastructure, bottlenecks in the use of charging stations are inevitable.
Despite the application of intelligent mechanisms for load distribution and the use of computer-aided processes for charging electric vehicles, also known as smart charging, scenarios such as the appropriate management of this type of infrastructure that facilitates reservations, receive little attention. 
Therefore, a system-controlled approach for the administration and distribution of charging facilities emerged as a key aspect for the further establishment of \acrfull{ev} in society and its acceptance as a safe and reliable means of transport.\\
% Target
What are the functional and non-functional requirements that such a system should satisfy? This paper explores this question from different perspectives utilizing detailed use cases based on a real-world scenario.
In cooperation with SAP SE, this master's thesis addresses this specific type of problem statement as an integral part of its subsequent work.\\
% Approach
The first part of this work begins with a theoretical examination of existing functionality covered by open industry standards, such as the \acrfull{ocpp} or \acrfull{ocpi}. 
As an open standard, many manufacturers of electric charging equipment implement these features as standard in their products, allowing a wide range of stations to be covered by using one of these for implementation.
Hereinafter, the collection and analysis of requirements that are not part of any of the already listed standards at the time of writing this thesis. 
Finally, a subset of the identified functional modules is used to implement the required basic functionality in the form of a \acrshort{poc}.\\
% Result
After the theoretical part, which isolates the necessary functionality for building a comprehensive reservation system, a basic set of operations and processes covering this functionality is implemented as part of the \acrshort{emobility} open source software from SAP SE, also known as \textit{Open e-Mobility}.
In order to validate the developed approach, the reservation module's implementation as independent functionality could be successfully validated using a simulated test environment, verifying the standards-compliant features incorporated by the web-based \href{https://github.com/JulianHBuecher/ev-dashboard/tree/reservation-process}{user interface}, the \href{https://github.com/JulianHBuecher/ev-mobile/tree/reservation-process}{mobile application} and the \href{https://github.com/JulianHBuecher/ev-server/tree/reservation-process}{backend service}.
Regarding current reservation systems, the suggested solution showcases the desired functionality of allowing users to oversee both allocated charging infrastructure as well as the corresponding reservations.\\
% Further Research Areas
Based on the underlying implementation, reservations can also be used in other scenarios in the described problem space. 
The use of reservations in combination with the time periods of the individual entries could be used for bi-directional charging, better known as \acrfull{v2g}, in order to avoid load peaks in the power grid.