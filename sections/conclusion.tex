%% LaTeX2e class for student theses
%% sections/conclusion.tex
%%
%% Karlsruhe University of Applied Sciences
%% Faculty of Computer Science and Business Information Systems
%%
%% --------------------------------------------------------
%% | Derived from sdqthesis by Erik Burger burger@kit.edu |
%% --------------------------------------------------------


\chapter{Resume and Outlook}
\label{ch:Resume and Outlook}

Regarding the results achieved as part of the previous chapters, the following sections provide a resume summarizing the essential parts of this work and providing an outlook considering the limitations of the elaborated approach and prepare possible problem statements for future work in this field.

\section{Resume}
\label{ch:Resume and Outlook:sec:Resume}

In retrospect, this work elaborates

\section{Outlook}
\label{ch:Resume and Outlook:sec:Outlook}

The usage of \acrshortpl{ev} and the requirement for an expanding charging infrastructure leads to necessity for further development in this area. Beside the basic functionalities provided today by different standards, a need for further development especially in the management of charging infrastructure is inevitable. 

\subsection{Future Work}
\label{ch:Resume and Outlook:sec:Outlook:ssec:Future Work}

Integration of Smart Charging service of the 'Open e-Mobility' solution as well as the simulation of the offered reservation approach utilizing tools like \href{https://github.com/aicenter/agentpolis}{AgentPolis}.
Furthermore, the notification regarding block vehicles on \acrshortpl{cs} parking lots could be integrated.
Additionally, optimization problems could be formulated based on the users preferences according the time, duration and charging power they require.
Reservations on mobile charging stations could be included as well.
As well as mapping to other existing open source standards to allow different kind of charging infrastructure to be registered.