%% LaTeX2e class for student theses
%% sections/abstracts/abstract_de.tex
%%
%% Karlsruhe University of Applied Sciences
%% Faculty of Computer Science and Business Information Systems
%%
%% --------------------------------------------------------
%% | Derived from sdqthesis by Erik Burger burger@kit.edu |
%% --------------------------------------------------------

\Abstract
% Problem
Aufgrund der sich stetig weiterentwickelnden Technologien im Kontext von Elektromobilit\"at und der damit verbundenen Auslastung von Lade-, wie auch Verkehrsinfrastruktur, sind Engp\"asse in der Verwendung von Ladestationen eine zu erwartende Konsequenz. 
Trotz der Nutzung intelligenter Mechanismen zur Lastverteilung und computer-gest\"utzter Verfahren f\"ur das Laden elektrisch-betriebener Fahrzeuge, beispielsweise Smart Charging, finden Szenarien wie das gezielte Verwalten der Nutzung vorhandener Ladeinfrastruktur in Form von Reservierungen nur wenig Verwendung.
Die daraus resultierende Notwendigkeit eines systemgest\"utzten Ansatzes zur Administration und Verteilung von Ladem\"oglichkeiten ist somit ein zentraler Aspekt f\"ur die weitere Etablierung von \acrfull{ev} innerhalb der Gesellschaft und deren Akzeptanz als sicheres und verl\"assliches Fortbewegungsmittel.\\
% Target
Welche funktionalen- und nicht-funktionalen Anforderungen sollte solch ein System erf\"ullen? Diese Frage wird anhand konkreter Use Cases, eingebettet in ein realitätsnahes Szenario, in der hier vorliegenden Arbeit aus unterschiedlichen Perspektiven beleuchtet. 
Die Bearbeitung dieser Problemstellung wurde in Form einer Master-Thesis in Kooperation mit der SAP SE durchgef\"uhrt.\\
% Approach
Beginnend mit einer theoretischen Betrachtung der bereits vorhandenen Funktionen, die basierend auf offenen Industriestandards, wie dem \acrfull{ocpp} oder dem \acrfull{ocpi}, durch die Hersteller elektrischer Lades\"aulen bereits angeboten werden, folgte die Sammlung und Analyse von Anforderungen, die w\"ahrend der Bearbeitungszeit der hier vorliegenden Arbeit noch nicht als Teil eines der bereits aufgeführten Standards abgedeckt wurden.
Anschließend erfolgte unter Verwendung einer Teilmenge der bereits identifizierten Funktionsbausteine die Umsetzung notwendiger Basisfunktionalität in Form eines \acrshort{poc}.\\
% Results
Resultierend konnte nach Durchlaufen der theoretischen und praktischen Phase dieser Arbeit eine Grundmenge an Operationen und Vorg\"angen auf Basis der Open Source Software \textbf{Open E-Mobility}, entwickelt von der SAP SE, beispielhaft als Bestandteil des konstruierten Szenarios umgesetzt und erfolgreich simuliert werden. 
Die Implementierung des Moduls für die Reservierung von Ladeinfrastruktur konnte als unabhängige Funktionalität innerhalb der unterschiedlichen Anwendungsbestandteile wie der webbasierten \href{https://github.com/JulianHBuecher/ev-dashboard}{Oberfläche}, der \href{https://github.com/JulianHBuecher/ev-mobile}{Mobile Applikaktion}, sowie des \href{https://github.com/JulianHBuecher/ev-server}{Backend-Servers} integriert werden.
In Anlehnung an bestehende Reservierungssysteme demonstriert die Anwendung, wie Nutzer \"uber das System zugewiesene Ladeinfrastruktur verwalten und f\"ur einen definierten Zeitraum reservieren k\"onnen.\\
% Further Research Areas
Entsprechend der zugrundegelegten Implementierung besteht die M\"oglichkeit der Nutzung von Reservierungen innerhalb anderer Szenarien, wie beispielsweise im Kontext von Smart Charging. Hier sind in Kombination mit den Zeitr\"aumen für die einzelnen Reservierungen Ans\"atze bez\"uglich bidirektionaler Ladevorg\"ange, beziehungsweise dem \acrfull{v2g}, zur Reduktion der Auslastung des Stromnetzes durchaus denkbar.