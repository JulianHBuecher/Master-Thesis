%% LaTeX2e class for student theses
%% sections/abstracts/abstract_de.tex
%%
%% Karlsruhe University of Applied Sciences
%% Faculty of Computer Science and Business Information Systems
%%
%% --------------------------------------------------------
%% | Derived from sdqthesis by Erik Burger burger@kit.edu |
%% --------------------------------------------------------

\Abstract
% Problem
Aufgrund der sich stetig weiterentwickelnden Technologien im Kontext von Elektromobilit\"at und der damit verbundenen Auslastung von Lade-, wie auch Verkehrsinfrastruktur, sind Engp\"asse in der Verwendung von Ladestationen eine zu erwartende Konsequenz. 
Trotz der Nutzung intelligenter Mechanismen zur Lastverteilung und computer-gest\"utzter Verfahren f\"ur das Laden elektrisch-betriebener Fahrzeuge, wie beispielsweise dem Smart Charging, finden Szenarien wie das gezielte Verwalten vorhandener Ladeinfrastruktur in Form von Reservierungen nur wenig Verwendung.
Die daraus resultierende Notwendigkeit eines systemgest\"utzten Ansatzes zur Administration und Verteilung von Ladem\"oglichkeiten ist somit ein zentraler Aspekt f\"ur die weitere Etablierung von elektrisch-betriebenen Fahrzeugen innerhalb der Gesellschaft und deren Akzeptanz als sicheres wie auch verl\"assliches Fortbewegungsmittel.\\
% Target
Welche funktionalen- und nicht-funktionalen Anforderungen solch ein System erf\"ullen sollte, wird als Frage anhand konkreter Use Cases, eingebettet in ein realitätsnahes Szenario, in der hier vorliegenden Arbeit aus unterschiedlichen Perspektiven beleuchtet.
Resultierend konnte die daraus extrahierte Problemstellung als Teil einer Kooperation mit der SAP SE in Form einer Master-Thesis bearbeitet werden.\\
% Approach
Zu Beginn stand eine theoretischen Betrachtung der bereits vorhandenen Funktionen, die basierend auf offenen Industriestandards, wie dem \acrfull{ocpp} oder dem \acrfull{ocpi}, durch die Hersteller elektrischer Lades\"aulen bereits angeboten werden.
Anhand dieser standardisierten Operationen, erfolgte anschließend eine Sammlung von Anforderungen, die w\"ahrend der Bearbeitungszeit der hier vorliegenden Arbeit noch nicht Teil einer der oben genannten Standards ber\"ucksichtigt wurden, jedoch essenziell f\"ur das zu erreichende Ziel identifiziert wurden.
Daraus resultierte wiederum durch Selektion einer Teilmenge notwendiger Basisfunktionalit\"at eine Umsetzung im Rahmen eines \acrshort{poc}.\\
% Results
Nach Durchlaufen der theoretischen und praktischen Phasen der hier vorliegenden Arbeit konnte eine Grundmenge an Operationen und Vorg\"angen synthetisiert und als Teil der SAP SE Open Source Software \textit{Open e-Mobility} beispielhaft unter Verwendung des konstruierten Szenarios innerhalb der etwaigen Anwendungen wie der webbasierten \href{https://github.com/JulianHBuecher/ev-dashboard/tree/reservation-process}{Oberfl\"ache}, der \href{https://github.com/JulianHBuecher/ev-mobile/tree/reservation-process}{mobilen Applikaktion}, sowie des \href{https://github.com/JulianHBuecher/ev-server/tree/reservation-process}{Backend-Servers} integriert und erfolgreich validiert werden.
In Anlehnung an bestehende Reservierungssysteme demonstriert dieser Ansatz eine Art und Weise, wie zuk\"unftige Nutzer \"uber das System zugewiesene Ladepunkte wie auch deren Reservierungen verwalten k\"onnen.\\
% Further Research Areas
Entsprechend der zugrundeliegenden Implementierung erlaubt das Konzept der Reservierung den Einsatz innerhalb weiterer Szenarien, wie beispielsweise im Kontext von Smart Charging. Hier sind in Kombination mit den Zeitr\"aumen für die einzelnen Reservierungen Ans\"atze bez\"uglich bidirektionaler Ladevorg\"ange, dem \acrfull{v2g}, zur Reduktion der Auslastung des Stromnetzes durchaus denkbar.
