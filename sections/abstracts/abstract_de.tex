%% LaTeX2e class for student theses
%% sections/abstract_de.tex
%%
%% Karlsruhe University of Applied Sciences
%% Faculty of  Computer Science and Business Information Systems
%%
%% --------------------------------------------------------
%% | Derived from sdqthesis by Erik Burger burger@kit.edu |
%% --------------------------------------------------------

\Abstract
% Problem
Aufgrund der sich stetig weiterentwickelnden Technologien im Kontext von Elektromobilit\"at und der daraus resultierenden steigenden Auslastung der existierenden Ladeinfrastruktur, sind Engp\"asse in der Nutzung \"offentlicher, wie auch nicht-\"offentlicher Ladestationen eine zu erwartende Konsequenz. 
Trotz der Evaluation und des Einsatzes von hoch-automatisierten Mechanismen zur Lastverteilung im Falle von Spannungsspitzen und der Nutzung optimierter Methodiken f\"ur das Laden elektrisch-betriebener Fahrzeuge, wie dem \Gls{smartcharging}, werden beispielsweise die Aspekte der geplanten Nutzung vorhandener Ladeinfrastruktur nur teilweise betrachtet. 
Die daraus resultierende Notwendigkeit eines systemgest\"utzten Ansatzes zur Administration und Verteilung von Ladem\"oglichkeiten ist somit ein zentraler Aspekt f\"ur die weitere Etablierung von elektrisch-betriebenen Fahrzeugen innerhalb der Gesellschaft.
% Target
Hieraus resultierte die zentrale Fragestellung nach den notwendigen Funktionen und Anforderungen an ein solches System, die in der hier vorliegenden Arbeit aus unterschiedlichen Perspektiven beleuchtet wurde. 
Diese Fragestellung wurde als Forschungsfrage einer Master-Thesis, die in Kooperation mit der SAP SE in der Abteilung S4 IC\&Q2C PM DHPR GER absolviert wurde, er\"ortert.
% Approach
Beginnend mit einer theoretischen Betrachtung der bereits vorhandenen Funktionen, die basierend auf Standards durch die Hersteller elektrischer Lades\"aulen bereits angeboten werden, folgte die Sammlung und Analyse von Anforderungen, die nicht durch etwaige Standards w\"ahrend der Bearbeitungszeit der hier vorliegenden Ausarbeitung abgedeckt wurden.
% Results
Somit konnte eine Grundmenge an Operationen und Vorg\"angen isoliert werden, die durch das zu konzeptionierende System in Form von  Basisfunktionalit\"at umzusetzen sind, um dem Nutzer eine m\"oglichst eing\"angige Verwendung zu gew\"ahrleisten.
Anschließend erfolgte die Umsetzung eines selktierten Feature Sets, die auf Basis der Open E-Mobility L\"osung, entwickelt von der SAP SE, beispielhaft mittels eines "Proof-of-Concept" umgesetzt wurde. 
Neben der noch fehlenden Implementierung, der im OCPP 1.6 umgesetzten ReserveNow und der Cancel Reservation Operation, die bisher noch nicht implementiert wurden, erfolgte die Implementierung der Reservierungs-Erweiterung \"uber die Anwendungsbestandteile des webbasierten \href{https://github.com/JulianHBuecher/ev-dashboard}{Dashboards}, der \href{https://github.com/JulianHBuecher/ev-mobile}{Mobile Applikaktion}, sowie des \href{https://github.com/JulianHBuecher/ev-server}{Backend-Servers}.
Die entstandene Anwendung soll in Anlehnung an bestehende Reservierungssysteme demonstrieren, wie Nutzer \"uber die unterschiedlichen Schnittstellen, die das System zur Verwaltung der Infrastruktur anbietet, mit geringem Aufwand die f\"ur sie passenden Optionen setzen und somit eine Ladestation f\"ur einen gewissen Zeitraum reservieren k\"onnen.
Eine Evaluation der implementierten Funktionen wurde anhand des \href{https://github.com/SAP/e-mobility-charging-stations-simulator}{e-mobility charging stations simulator} durchgef\"uhrt, der entsprechend der bereits angesprochenen fehlenden Funktionen aus dem OCPP 1.6 im Rahmen dieser Arbeit angepasst wurde.
% Further Research Areas
Eine Integration der Funktionalit\"aten des \href{https://github.com/sap-labs-france/emobility-smart-charging}{Smart Charging Services}, der als Teil der oben genannten Open Source L\"osung ist, waren dabei nicht Teil der Betrachtung. Es werden jedoch Ans\"atze und M\"oglichkeiten beschrieben, die innerhalb weiterer Arbeiten verwendet werden k\"onnen, um Reservierungen und \Gls{smartcharging} zu verbinden.
